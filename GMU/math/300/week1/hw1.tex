\documentclass{article}
\usepackage{amsfonts}
\begin{document}
  MATH-300 \hfill Andrew Jones
  \centering
  \section{Homework}
  Prove or Disprove the following statements: \newline
  \newline
  \raggedright
  1. \(\exists\) n \(\in \mathbb{Z}\): n + 1 = 5
  \begin{itemize}
    \item There exists a number n in the integers such that n + 1 = 5
    \item Set n equal to 4
    \item Observe that 4 + 1 = 5 \(\in \mathbb{Z}\)
  \end{itemize}
  2. \(\forall\) n \(\in \mathbb{Z}\): \(n > 7\)
  \begin{itemize}
    \item For all numbers n in the integers, n is greater than 7
    \item Set n equal to 5
    \item Observe that 5 \(\in \mathbb{Z}\)
  \end{itemize}
  3. \(\exists\) x \(\in \mathbb{R}: \forall\) y \(\in \mathbb{R}: x \geq y \)
  \begin{itemize}
    \item There exists a real number x such that for all real numbers y: n \(\geq\) y
    \item Fix x to 5.
    \item Fix y to 7.
    \item Observe that 5, 7 \(\in \mathbb{R}\)
    \item Observe that y \(\ge\) n where y = 7 and x = 5
  \end{itemize}
  4. \(\exists\) x \(\in \mathbb{R}\): \(\forall\) k \(\in \mathbb{N}\): \(x^k = x \) 
  \begin{itemize}
    \item There exists a real number x such that for all integers y: \(x^k = x \) 
    \item Fix x to 1.
    \item Observe that \(1^y = 1\)
  \end{itemize}
  5. \(\forall\) x \(\in \mathbb{R}\): \(\exists\) y \(\in \mathbb{R}\): xy = 1
  \begin{itemize}
    \item For all x in the reals there exists a real number y such that xy = 1.
    \item Put \(y = \frac{1}{x}\)
    \item Observe that \(x * \frac{1}{x} = 1\)
    \item \(\frac{1}{x}\) \(\in \mathbb{R}\)
  \end{itemize}
  6. \(\exists\) x \(\in \mathbb{R}\): \(\forall\) y \(\in \mathbb{R}\): xy = y
  \begin{itemize}
    \item There exists a real number x such that for all real numbers xy = y.
    \item Put x = 1
    \item Observe that 1*y=y
    \item 1 \(\in \mathbb{R}\)
  \end{itemize}
  7. Give an example of a proposition P for which:
  \begin{center}
  \(\forall\) m \(\in \mathbb{Z}\): \(\exists\) n \(\in \mathbb{Z}\): P(m,n) is true
  and
  \(\exists\) n \(\in \mathbb{Z}\): \(\forall\) m \(\in \mathbb{Z}\): P(m,n) is false
  \end{center}
  \begin{itemize}
    \item For all integers m there exists an integer n such that P(m,n) is true
    \item There exists an integer n such that for all integers m P(m,n) is false
    \item Let P = m = n
    \item Put n = m
    \item Observe that n, m \(\in \mathbb{Z}\)
    \item Observe that for all integers m there exists an integer n such that m = n
    \item Observe that there is not a single integer n such that for all integers m m = n
  \end{itemize}
  8. Find a Proposition Q for which:
  \begin{center}
  \(\forall\) m \(\in \mathbb{Z}\): \(\exists\) n \(\in \mathbb{Z}\): Q(m,n) is false
  and
  \(\exists\) n \(\in \mathbb{Z}\): \(\forall\) m \(\in \mathbb{Z}\): Q(m,n) is true
  \end{center}
  \begin{itemize}
    \item For all integers m there exists an integer n such that Q(m,n) is false
    \item There exists an integer n such that for all integers m Q(m,n) is true
    \item Put Q equal to m < n
    \item Put n = (m + 1)
    \item Observe that m, (m + 1) \(in \mathbb{Z}\)
    \item Observe that for all integers m there is an integer m - 1 such that m < (m + 1)
    \item Put m = 5 and n = 4
    \item Observe that 5 < 4 is false
    \item Observe that there does not exist a single integer n such that all integers m are greater than it.
  \end{itemize}
  9. Is the statement \(\forall\) a \(\in \mathbb{A}\): \(\forall\) b \(\in \mathbb{B}\): P(a,b) communative and there for \(\forall\) b \(\in \mathbb{B}\): \(\forall\) a \(\in \mathbb{A}\): P(a,b) is also true? 
  \newline
  \begin{itemize}
    \item For all numbers a in A such that for all numbers b in B satisfy P(a,b)
    \item For all numbers b in B such that for all numbers a in A satisfy P(a,b)
    \item As the order of the arguments to the proposition does not change, I would assume that switching the order of the for all statements should not effect the value of the proposition.
  \end{itemize}
  10. \(\exists\) a \(\in \mathbb{A}\): \(\exists\) b \(\in \mathbb{B}\): P(a,b) does it follow that \(\exists\) b \(\in \mathbb{B}\): \(\exists\) a \(\in \mathbb{A}\): P(a,b)? 
  \begin{itemize}
    \item There exists a number a in Set A such that there exists a number B in set B that satisifies P(a,b)
    \item There exists a number b in Set B such that there exists a number A in Set A that satisfies P(a,b)
    \item I would also assume here that exists is communative and what matters is switching the order of the parameters to the Property.
  \end{itemize}
\end{document}
