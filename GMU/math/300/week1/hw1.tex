\documentclass{article}
\usepackage{amsfonts}
\begin{document}
  MATH-300 \hfill Andrew Jones
  \centering
  \section{Homework}
  Prove or Disprove the following statements: \newline
  \newline
  \raggedright
  1. $\exists n \in \mathbb{Z}: n + 1 = 5$
  \begin{flushleft}
    There exists a number n in the integers such that n + 1 = 5.
    Set n equal to 4.
    Observe that $4 + 1 = 5 \in \mathbb{Z}$.
  \end{flushleft}
  2. $\forall n \in \mathbb{Z}: n > 7$
  \begin{flushleft}
    For all numbers n in the integers, n is greater than 7.
    Set n equal to 5.
    Observe that $5 \in \mathbb{Z}$.
  \end{flushleft}
  3. $\exists x \in \mathbb{R}: \forall y \in \mathbb{R}: x \geq y$
  \begin{flushleft}
    There exists a real number x such that for all real numbers $y: n \geq y$.
    Fix x to 5.
    Fix y to 7.
    Observe that 5, 7 $\in \mathbb{R}$.
    Observe that $y \ge n $ where y = 7 and x = 5.
  \end{flushleft}
  4. $\exists x \in \mathbb{R}: \forall k \in \mathbb{N}: x^k = x$ 
  \begin{flushleft}
    There exists a real number x such that for all integers y: $x^k = x$. 
    Fix x to 1.
    Observe that $1^y = 1$.
    There for the proposition is true.
  \end{flushleft}
  5. $\forall x \in \mathbb{R}: \exists y \in \mathbb{R}: xy = 1$
  \begin{flushleft}
    For all x in the reals there exists a real number y such that xy = 1.
    Put $y = \frac{1}{x}$.
    Observe that $x * \frac{1}{x} = 1$.
    $\frac{1}{x} \in \mathbb{R}$.
    Therefore the proposition is true.
  \end{flushleft}
  6. $\exists x \in \mathbb{R}: \forall y \in \mathbb{R}: xy = y$
  \begin{flushleft}
    There exists a real number x such that for all real numbers xy = y. Put x = 1
    Observe that 1*y=y. 1 $\in \mathbb{R}$.
  \end{flushleft}
  7. Give an example of a proposition P for which:
  \begin{center}
  $\forall m \in \mathbb{Z}: \exists n \in \mathbb{Z}: P(m,n)$ is true
  and
  $\exists n \in \mathbb{Z}: \forall m \in \mathbb{Z}: P(m,n)$ is false
  \end{center}
  \begin{flushleft}
    For all integers m there exists an integer n such that P(m,n) is true. There exists an integer n such that for all integers m P(m,n) is false. Let P = m = n. Put n = m. Observe that n, m $\in \mathbb{Z}$. Observe that for all integers m there exists an integer n such that m = n. Observe that there is not a single integer n such that for all integers m m = n
  \end{flushleft}
  8. Find a Proposition Q for which:
  \begin{center}
  $\forall m \in \mathbb{Z}: \exists n \in \mathbb{Z}: Q(m,n)$ is false
  and
  $\exists n \in \mathbb{Z}: \forall m \in \mathbb{Z}: Q(m,n)$ is true
  \end{center}
  \begin{flushleft}
    For all integers m there exists an integer n such that Q(m,n) is false. There exists an integer n such that for all integers m Q(m,n) is true. Put Q equal to m < n. Put n = (m + 1). Observe that m, (m + 1) $in \mathbb{Z}$. Observe that for all integers m there is an integer m - 1 such that m < (m + 1). Put m = 5 and n = 4. Observe that $5 < 4$ is false. Observe that there does not exist a single integer n such that all integers m are greater than it.
  \end{flushleft}
  9. Is the statement $\forall a \in \mathbb{A}: \forall b \in \mathbb{B}: P(a,b)$ communative and there for $\forall b \in \mathbb{B}: \forall a \in \mathbb{A}: P(a,b)$ is also true? 
  \begin{flushleft}
    For all numbers a in A such that for all numbers b in B satisfy P(a,b). For all numbers b in B such that for all numbers a in A satisfy P(a,b). As the order of the arguments to the proposition does not change, I would assume that switching the order of the for all statements should not effect the value of the proposition.
  \end{flushleft}
  10. $\exists a \in \mathbb{A}: \exists b \in \mathbb{B}: P(a,b)$ does it follow that $\exists b \in \mathbb{B}: \exists a \in \mathbb{A}: P(a,b)$? 
  \begin{flushleft}
  There exists a number a in Set A such that there exists a number B in set B that satisifies P(a,b). There exists a number b in Set B such that there exists a number A in Set A that satisfies P(a,b). I would also assume here that exists is communative and what matters is switching the order of the parameters to the Property.
  \end{flushleft}
\end{document}
