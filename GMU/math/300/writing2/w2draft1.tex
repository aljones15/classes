\documentclass{article}
\usepackage{amsmath,amssymb,amsfonts,amsthm}
\usepackage[shortlabels]{enumitem}
\usepackage{fullpage}

% mathematical environments - see https://en.wikibooks.org/wiki/LaTeX/Theorems

\theoremstyle{plain}
\newtheorem{lemma}{Lemma}
\newtheorem{proposition}{Proposition}
\newtheorem{theorem}{Theorem}
\newtheorem{corollary}{Corollary}

\theoremstyle{definition}
\newtheorem{definition}{Definition}

\theoremstyle{remark}
\newtheorem{remark}{Remark}
\newtheorem{example}{Example}

% commands

\newcommand{\N}{\mathbb{N}}
\newcommand{\Z}{\mathbb{Z}}
\newcommand{\Q}{\mathbb{Q}}
\newcommand{\R}{\mathbb{R}}

\renewcommand{\emptyset}{\varnothing}
\renewcommand{\subset}{\subseteq}
\renewcommand{\supset}{\supseteq}
\renewcommand{\d}{\mathrm{d}}



\title{Category Theory Paper}
\author{Andrew L Jones}\date{}

\begin{document}\maketitle


% INTRODCUTION

\section*{Introduction}

Note that
\begin{enumerate}[i.,noitemsep]
	\item first point
	\item second point
	\item third point
\end{enumerate}



% FIRST SECTION

\section{First section}

\begin{definition}
	A \emph{number} is a\textellipsis
\end{definition}



% SECOND SECTION

\section{Second section}

\begin{lemma}
	We have
	\[
		\int_0^\pi \sin(3x)\,\d x = \frac{2}{3}.
	\]
\end{lemma}

\begin{proof}
	A direct computation yields
	\begin{align*}
		\int_0^\pi \sin(3x)\,\d x
			&=	\frac{1}{3}\int_0^{3\pi} \sin u \,\d u,		\hspace{1cm} u=3x,		\\
			&=	\frac{1}{3} \big[-\cos u\big]_0^{3\pi}						\\
			&=	\frac{1}{3} \big[1-(-1)\big]						\\
			&=	\frac{2}{3}.
	\end{align*}
\end{proof}

\begin{remark}
	This is interesting since\textellipsis
\end{remark}
\end{document}
