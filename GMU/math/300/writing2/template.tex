\documentclass{article}
\usepackage{math}

\title{An Introduction to Category Theory}
\author{Andrew L Jones}\date{}

\begin{document}
\maketitle


% INTRODUCTION

\section*{Introduction}


% FIRST SECTION 

\section{Objects, Morphisms, Categories}

Lorem ipsum dolor sit amet, consectetur adipisci tempor incidunt ut labore et dolore magna aliqua veniam, quis nostrud exercitation ullamcorpor s commodo consequat. 

\begin{definition}\label{def:lorem}
	Lorem ipsum dolor sit amet, consectetur adipisci tempor incidunt ut labore et dolore magna aliqua veniam, quis nostrud exercitation ullamcorpor s commodo consequat.
\end{definition}

Lorem ipsum dolor sit amet, consectetur adipisci tempor incidunt ut labore et dolore magna aliqua veniam, quis nostrud exercitation ullamcorpor s commodo consequat. 

\begin{theorem}\label{thm:ipsum}
	Duis autem vel eum irrure esse molestiae consequat, vel illum dolore eu fugi et iusto odio dignissim qui blandit praesent luptat exceptur sint occaecat cupiditat non provident, deserunt mollit anim id est laborum et dolor fuga distinct.
\end{theorem}


% SECOND SECTION

\section{Functors and Natural Transformations}

Applying Theorem \ref{thm:ipsum} to Definition \ref{def:lorem}, it follows that ipsum. This was previously established in \cite{Gauss86} and \cite{Euclid02,Euclid03}.

\begin{theorem}
X is a functor
\end{theorem}
\begin{proof}
Proof that X is a functor
\end{proof}



% BIBLIOGRAPHY

\bibliographystyle{plain}  % see:  https://www.overleaf.com/learn/latex/Bibtex_bibliography_styles
\bibliography{template}




\end{document}





































