\documentclass{article}
\usepackage{math}

\title{An Introduction to Category Theory}
\author{Andrew L Jones}\date{}

\begin{document}
\maketitle


% INTRODUCTION

\section*{Introduction}
The Roman numerals lack a zero and therefore are not abstract enough to signify nothing. The Roman numbers came from counting physical objects, zero came from questions about how to form counting systems. Interestingly, one of math's most abstract systems Category Theory arose from the habit of representing relations as diagrams on blackboards. While it's origins might be in the corporeal world of chalkboards and erasers, Category Theory is a field of mathematics that continues a long tradition to study math in the abstract.

% FIRST SECTION 

\section{Objects and Arrows}
Arrows can be and usually are functions. Objects can be and usually are sets.
\begin{definition}
Definition for Category
\end{defintion}

\begin{example}
Use the Monoid A(Z, +) here.
\end{example}

Lorem ipsum dolor sit amet, consectetur adipisci tempor incidunt ut labore et dolore magna aliqua veniam, quis nostrud exercitation ullamcorpor s commodo consequat. 

\begin{definition}\label{def:lorem}
	Lorem ipsum dolor sit amet, consectetur adipisci tempor incidunt ut labore et dolore magna aliqua veniam, quis nostrud exercitation ullamcorpor s commodo consequat.
\end{definition}

Lorem ipsum dolor sit amet, consectetur adipisci tempor incidunt ut labore et dolore magna aliqua veniam, quis nostrud exercitation ullamcorpor s commodo consequat. 

\begin{theorem}\label{thm:ipsum}
	Duis autem vel eum irrure esse molestiae consequat, vel illum dolore eu fugi et iusto odio dignissim qui blandit praesent luptat exceptur sint occaecat cupiditat non provident, deserunt mollit anim id est laborum et dolor fuga distinct.
\end{theorem}


% SECOND SECTION

\section{Functors and Natural Transformations}

Applying Theorem \ref{thm:ipsum} to Definition \ref{def:lorem}, it follows that ipsum. This was previously established in \cite{Gauss86} and \cite{Euclid02,Euclid03}.

\begin{theorem}
X is a functor
\end{theorem}
\begin{proof}
Proof that X is a functor
\end{proof}



% BIBLIOGRAPHY

\bibliographystyle{plain}  % see:  https://www.overleaf.com/learn/latex/Bibtex_bibliography_styles
\bibliography{template}




\end{document}





































