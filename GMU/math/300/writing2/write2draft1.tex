\documentclass{article}
\usepackage{math}

\title{An Introduction to Category Theory}
\author{Andrew L Jones}\date{}

\begin{document}
\maketitle


% INTRODUCTION

\section*{Introduction}
One of math's most abstract fields: Category Theory arose from the habit of representing relations as diagrams on blackboards. While it's origins might be in the corporeal world of chalkboards and erasers, Category Theory is a field of mathematics that ephaises the abstract study of mathematics as form and relation over the applied use of mathematics as calculation.

% FIRST SECTION 

\section{Objects and Arrows}
Fundamental to Category Theory are categories.
\begin{definition}
A category consists of a class of Objects ob(c); a class mor(C) of Arrows; a source of Objects to map from; a target of Objects to map to. Categories must satisfy three conditions:
  \begin{enumerate}
    \item Arrows must be associative
    \item Arrows most composed with other Arrows
    \item All Objects must have a left and right identity that is part of the Arrows
  \end{enumerate}
\end{definition}
Arrows can be and usually are functions. Objects can be and usually are sets. Both arrows and objects are members of classes which allows Category Theory to abstract over collections that are often problematic to define in terms of their objects. Hence Category Theory can reason about large sets resulting in a field that is more about structure than content. “Category theory is extreme in the sense that it actively discourages us from looking inside the objects. An object in category theory is an abstract nebulous entity.” - Page 10 - Category Theory for Programmers - Bartosz Milewski.

\begin{theorem}
  The set of all sets is a category
\end{theorem}

\begin{proof}
Let $\Omega$ be the set of all sets. Observe that $\Omega$ is a class containing Objects $U(\Omega)$. Let $Id_{\Omega}$ be $f: A \to A$.
\end{proof}

\begin{example}
Use the Monoid A(Z, +) here.
\end{example}

Lorem ipsum dolor sit amet, consectetur adipisci tempor incidunt ut labore et dolore magna aliqua veniam, quis nostrud exercitation ullamcorpor s commodo consequat. 

\begin{definition}\label{def:lorem}
	Lorem ipsum dolor sit amet, consectetur adipisci tempor incidunt ut labore et dolore magna aliqua veniam, quis nostrud exercitation ullamcorpor s commodo consequat.
\end{definition}

Lorem ipsum dolor sit amet, consectetur adipisci tempor incidunt ut labore et dolore magna aliqua veniam, quis nostrud exercitation ullamcorpor s commodo consequat. 

\begin{theorem}\label{thm:ipsum}
	Duis autem vel eum irrure esse molestiae consequat, vel illum dolore eu fugi et iusto odio dignissim qui blandit praesent luptat exceptur sint occaecat cupiditat non provident, deserunt mollit anim id est laborum et dolor fuga distinct.
\end{theorem}


% SECOND SECTION

\section{Functors and Natural Transformations}

Applying Theorem \ref{thm:ipsum} to Definition \ref{def:lorem}, it follows that ipsum. This was previously established in \cite{Gauss86} and \cite{Euclid02,Euclid03}.

\begin{theorem}
X is a functor
\end{theorem}
\begin{proof}
Proof that X is a functor
\end{proof}



% BIBLIOGRAPHY

\bibliographystyle{plain}  % see:  https://www.overleaf.com/learn/latex/Bibtex_bibliography_styles
\bibliography{template}




\end{document}





































