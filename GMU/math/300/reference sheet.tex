\documentclass{article}
%\usepackage{fullpage}
\usepackage[left=2cm,right=2cm,top=3cm,bottom=2.5cm]{geometry}
\usepackage{amsmath,amsthm,amssymb}
\usepackage[shortlabels]{enumitem}

\theoremstyle{definition}
\newtheorem{definition}{Definition}

\renewcommand{\subset}{\subseteq}
%\setlength{\leftmargini}{0pt}


\newcommand{\C}{\mathbb{C}}
\newcommand{\R}{\mathbb{R}}
\newcommand{\Q}{\mathbb{Q}}
\newcommand{\N}{\mathbb{N}}
\newcommand{\Z}{\mathbb{Z}}

\newcommand{\id}{\mathrm{id}}
\renewcommand{\d}{\mathrm{d}}


\title{\emph{\Large Final Exam} \\ Reference Sheet}
\author{}\date{}



\begin{document}\maketitle


\section*{Set operations and functions}

\begin{definition}
	The \emph{empty set} $\emptyset$ is the set that contains no elements. 
\end{definition}

\begin{definition}
	The \emph{union} of $A$ and $B$ is
	\[
		A\cup B = \big\{x \,|\, x\in A \text{ or }x\in B\big\}.
	\]
\end{definition}

\begin{definition}
	The \emph{intersection} of $A$ and $B$ is
	\[
		A\cap B = \big\{x \,|\, x\in A \text{ and }x\in B\big\}.
	\]
\end{definition}

\begin{definition}
	We say that $A$ and $B$ are \emph{disjoint} when $A\cap B=\emptyset$.
\end{definition}

\begin{definition}
	We say that $A$ is a \emph{subset} of $B$ if
	\[
		\forall x : \big(x\in A \to x\in B\big).
	\]
	In this case, we write $A\subset B$.
\end{definition}

\begin{definition}
	The \emph{difference} of $A$ and $B$ is
	\[
		B\backslash A = \{x\in B \,|\, x\notin A\}.
	\]
\end{definition}

\begin{definition}
	The \emph{symmetric difference} of $A$ and $B$ is $A\,\Delta\, B = A\backslash B \cup B\backslash A$.
\end{definition}

\begin{definition}
	If $A\subset B$, then the \emph{complement} of $A$ in $B$ is $A^c=B\backslash A$.
\end{definition}

\begin{definition}
	The \emph{composition} of $f:A\to B$ and $g:B\to C$ is
	\begin{align*}
		g\circ f:	A	&\to		C	\\
				x	&\mapsto	g(f(x)).
	\end{align*}
\end{definition}


\section*{Injective and surjective functions}

\begin{definition}
	The function $f:A\to B$ is said to be \emph{injective} if $f(x)=f(y)$ implies $x=y$.
\end{definition}

\begin{definition}
	The function $f:A\to B$ is called \emph{surjective} when for every $y\in B$ there is an $x\in A$ with $f(x)=y$.
\end{definition}

\begin{definition}
	We say that $f:A\to B$ is \emph{bijective} when it is both injective and surjective.
\end{definition}

\begin{definition}
	The \emph{restriction} of $f:A\to B$ to $S$ is the function
	\begin{align*}
		f|_S:	S	&\to		\;B	\\
			x	&\mapsto	f(x).
	\end{align*}
\end{definition}


\section*{Limits}

\begin{definition}
	We say that $f(x)\to\infty$ as $x\to\infty$, or that $\displaystyle\lim_{x\to\infty}f(x)=\infty$, when
	\[
		\forall M>0 : \exists N>0 : \forall x>N : f(x)>M.
	\]
	In this case, we write $\displaystyle\lim_{x\to\infty}f(x)=\infty$.
\end{definition}

\begin{definition}
	Fix $L\in\R$. We say that $f(x)\to L$ as $x\to\infty$, or that $\displaystyle\lim_{x\to\infty}f(x)=L$, when
	\[
		\forall \epsilon>0 : \exists N>0 : \forall x>N : |f(x)-L|<\epsilon.
	\]
\end{definition}

\begin{definition}
	Fix $x_0\in\R$. We say that $f(x)\to\infty$ as $x\to x_0$, or that $\displaystyle\lim_{x\to x_0}f(x)=\infty$, when
	\[
		\forall M>0 : \exists \delta>0 : \forall x\in\R : |x-x_0|<\delta \implies f(x)>M.
	\]
\end{definition}

\begin{definition}
	Fix $x_0\in\R$ and $L\in\R$. We say that $f(x)\to L$ as $x\to x_0$, or that $\displaystyle\lim_{x\to\infty}f(x)=L$, when
	\[
		\forall\epsilon>0 : \exists \delta>0 : \forall x\in\R : |x-x_0|<\delta \implies |f(x)-L|<\epsilon.
	\]
\end{definition}


\section*{Relations}

\begin{definition}
	The \emph{Cartesian product} of $A$ and $B$ is the set of ordered pairs
	\[
		A\times B = \{(a,b)\mid a\in A, b\in B\}.
	\]
\end{definition}

\begin{definition}
	A \emph{relation} from $A$ to $B$ is a subset $R\subset A\times B$.
\end{definition}

\begin{definition}
	The \emph{domain} of $R\subset A\times B$ is the subset
	\[
		\mathrm{Dom}\, R = \{a\in A \mid \exists b\in B : aRb\},
	\]
	and the \emph{range} is
	\[
		\mathrm{Rng}\, R = \{b\in B \mid \exists a\in A : aRb\}.
	\]
\end{definition}

\begin{definition}
	A \emph{function} from $A$ to $B$ is a relation $f\subset A\times B$ such that
	\[
		\forall a\in A : \exists!\, b\in B : (a,b)\in f.
	\]
	We usually write $(a,b)\in f$ as $f(a)=b$.
\end{definition}

\begin{definition}
	If $R\subset A\times B$ is a relation from $A$ to $B$, then the \emph{inverse} of $R$ is the relation $R^{-1}\subset B\times A$ given by
	\[
		bR^{-1}a \iff aRb.
	\]
\end{definition}

\begin{definition}
	If $R$ is a relation from $A$ to $B$, and if $S$ is a relation from $B$ to $C$, then the \emph{composition} of $R$ and $S$ is
	\[
		S\circ R = \{(a,c) \mid \exists b\in B : aRb \:\wedge\: bSc\}.
	\]
\end{definition}


\section*{Equivalence relations and partial orders}

\begin{definition}
	We say that the relation $R$ on $A$ is
	\begin{itemize}
		\item \emph{reflexive} when $\forall a\in A : aRa$
		\item \emph{transitive} when $\forall a,b,c\in A : (aRb\wedge bRc) \implies aRc$
		\item \emph{symmetric} when $\forall a,b\in A : aRb \implies bRa$
		\item \emph{antisymmetric} when $\forall a,b\in A : (aRb \wedge bRa) \implies a=b$
	\end{itemize}
\end{definition}

\begin{definition}
	The relation $R$ on $A$ is
	\begin{itemize}
		\item an \emph{equivalence relation} when it is reflexive, transitive, and symmetric;
		\item a \emph{partial order} when it is reflexive, transitive, and antisymmetric.
	\end{itemize}
\end{definition}

\begin{definition}
	A set $A$ equipped with a partial order $R$ is called a \emph{partially ordered set} or a \emph{poset}.
\end{definition}

\begin{definition}
	The \emph{quotient map} associated to a set $A$ with equivalence relation $\sim$ is the function
	\begin{align*}
		q:	A	&\to		A/\!\sim	\\
			a	&\mapsto	[a].
	\end{align*}
\end{definition}

\section*{Number systems}

\begin{definition}
	The \emph{complex numbers} $(\C,+,\times)$ comprise
	\begin{enumerate}
		\item the set $\C=\R^2$,
		\item the binary operation
			\begin{align*}
				+:	\hspace{.8cm}\C^2\hspace{.8cm}	&\longrightarrow	\hspace{.8cm}\C		\\
					((a,b),(a',b'))			&\longmapsto		(a+a',b+b'),
			\end{align*}
		\item the binary operation
			\begin{align*}
				\times:	\hspace{.8cm}\C^2\hspace{.8cm}	&\longrightarrow	\hspace{1.2cm}\C	\\
					((a,b),(a',b'))			&\longmapsto		(aa'-bb',ab'+a'b).
			\end{align*}
	\end{enumerate}
\end{definition}

\begin{definition}
	The \emph{rational numbers} $(\Q,+,\times)$ consist of
	\begin{enumerate}
		\item the set $\Q=\{(p,q) \in\Z^2 \mid q\neq 0\}\,/\!\sim$,
		\item the binary operation
			\begin{align*}
				+:	\hspace{1.5cm}\Q^2\hspace{.8cm}	&\longrightarrow	\hspace{.8cm}\Q		\\
					\big([(p,q)],[(p',q')]\big)	&\longmapsto		\big[(pq'+p'q,qq')\big],
			\end{align*}
		\item the binary operation
			\begin{align*}
				\times:	\hspace{1.2cm}\Q^2\hspace{1.0cm}	&\longrightarrow	\hspace{.8cm}\Q	\\
					\big([(p,q),(p',q')]\big)		&\longmapsto		\big[(pp',qq')\big].
			\end{align*}
	\end{enumerate}
\end{definition}

\begin{definition}
	The \emph{integers} $(\Z,+,\times)$ consist of
	\begin{enumerate}[i.]
		\item the set $\Z=\N^2$,
		\item the binary operation
			\begin{align*}
				+:	\hspace{.8cm}\Z^2\hspace{.8cm}	&\longrightarrow	\hspace{.8cm}\Z		\\
					((m,n),(m',n'))			&\longmapsto		(m+m',n+n'),
			\end{align*}
		\item the binary operation
			\begin{align*}
				\times:	\hspace{1.0cm}\Z^2\hspace{.8cm}	&\longrightarrow	\hspace{1.2cm}\Z	\\
					((m,n),(m',n'))			&\longmapsto		(mm'+nn',mn'+m'n).
			\end{align*}
	\end{enumerate}
\end{definition}


\section*{Algebraic structures with one binary operation}

\begin{definition}
	A \emph{binary operation} on $A$ is a function
	\[
		* : A\times A \to A.
	\]
\end{definition}

\begin{definition}
	A \emph{magma} $(A,*)$ is a set $A$ equipped with a binary operation $*:A\times A\to A$.
\end{definition}

\begin{definition}
	We say that $*$ is
	\begin{itemize}
		\item \emph{commutative} when
			\[
				\forall a,b\in A : a*b = b*a
			\]
		\item \emph{associative} when
			\[
				\forall a,b,c\in A : (a*b)*c = a*(b*c)
			\]
	\end{itemize}
\end{definition}

\begin{definition}
	We say that $(A,*)$ is a \emph{semigroup} when $*$ is associative. If $*$ is additionally commutative, then $(A,*)$ is called a \emph{commutative semigroup}.
\end{definition}

\begin{definition}
	We say that $e\in A$ is an \emph{identity element} for $*:A\times A\to A$ when
	\[
		\forall a\in A : a*e = a = e*a.
	\]
\end{definition}

\begin{definition}
	A semigroup $(A,*)$ that admits an identity element $e\in A$ is called a \emph{monoid}.
\end{definition}

\begin{definition}
	Fix an element $a\in A$. If $b\in A$ satisfies
	\[
		a*b = e = b*a
	\]
	then $b$ is called an \emph{inverse element} of $a$, and we write $b=a^{-1}$.
\end{definition}

\begin{definition}
	A semigroup $(A,*)$ is called a \emph{group} when every $a\in A$ has an inverse $a^{-1}\in A$.
\end{definition}

\begin{definition}
	A group $(A,*)$ is an \emph{abelian group} when $*$ is commutative.
\end{definition}


\section*{Algebraic structures with multiple binary operations}

\begin{definition}
	A \emph{ring} $(R,+,\cdot)$ comprises a set $R$ and two binary operations $+,\cdot : R\times R\to R$, such that
	\begin{enumerate}[i.]
		\item $(R,+)$ is an abelian group,
		\item $(R,\cdot)$ is a monoid,
		\item the operation $\cdot$ \emph{distributes} over $+$, that is, for all $a,b,c\in R$,
			\begin{align*}
				a\cdot(b+c)	&=	(a\cdot b)+(a\cdot c)		\\
				(a+b)\cdot c	&=	(a\cdot c)+(b\cdot c)
			\end{align*}
	\end{enumerate}
\end{definition}

\begin{definition}
	A \emph{zero divisor} in a commutative ring $(A,+,\cdot)$ is an element $a\in A$ for which there exists a nonzero $b\in A$ with $ab=0$.
\end{definition}

\begin{definition}
	A commutative ring $(R,+,\cdot)$ is called an \emph{integral domain} when
	\begin{enumerate}[i.]
		\item it does not contain any nonzero zero divisor,
		\item $0\neq 1$.
	\end{enumerate}
\end{definition}

\begin{definition}
	An integral domain $(R,+,\cdot)$ is called a \emph{field} when every nonzero element $a\in R\backslash\{0\}$ has a multiplicative inverse $a^{-1}\in R$,
\end{definition}

\begin{definition}
	A \emph{$k$-vector space} $(V,+,\cdot)$ comprises a set $V$ together with operations
	\[
		+ : V\times V \to V
	\]
	and
	\[
		\cdot : k\times V \to V
	\]
	such that
	\begin{enumerate}[i.]
		\item $(V,+)$ is an abelian group
		\item \emph{scalar multiplication} $\cdot$ and \emph{vector addition} $+$ satisfy
			\begin{align*}
				1 \cdot u		&=	u			\\
				(a+b)\cdot u		&=	(a\cdot u) + (b\cdot u)	\\
				a\cdot(b\cdot u)	&=	(a\cdot b)\cdot u	\\
				a\cdot(u+v)		&=	(a\cdot u)+(a\cdot v)
			\end{align*}
	\end{enumerate}
\end{definition}

\begin{definition}
	An \emph{$R$-module} $(V,+,\cdot)$ comprises a set $V$ together with operations
	\[
		+ : V\times V \to V
	\]
	and
	\[
		\cdot : k\times V \to V
	\]
	that together satisfy the familiar vector space conditions.
\end{definition}


\section*{Homomorphisms}

\begin{definition}
	A homomorphism $f:X\to Y$ is called a
	\begin{enumerate}[i.]
		\item \emph{monomorphism} if
			\[
				\forall \,(g,g' : Z\to X) : (f\circ g = f\circ g') \implies g=g',
			\]
			that is, $f$ is \emph{left-cancellative},
		\item \emph{epimorphism} if
			\[
				\exists \,(h : Y\to X) : (h\circ f = h'\circ f) \implies h=h',
			\]
			that is, $f$ is \emph{right-cancellative},
		\item \emph{isomorphism} if
			\[
				\exists \,(k : Y\to X) : (f\circ k = \id_Y) \:\wedge\: (k\circ f=\id_X),
			\]
			that is, $f$ has an \emph{inverse} $k$.
	\end{enumerate}
\end{definition}

\begin{definition}
	\begin{enumerate}[i.]
		\item A homomorphism $f:X\to X$ is called an \emph{endomorphism}.
		\item An isomorphism $f:X\to X$ is called an \emph{automorphism}.
	\end{enumerate}
\end{definition}

\begin{definition}
	We say that $X$ and $Y$ are \emph{isomorphic} if there exists an isomorphism $f:X\xrightarrow{\sim}Y$.
\end{definition}

\begin{definition}
	A \emph{group homomorphism} from $(G,\cdot)$ to $(H,*)$ is a function $f:G\to H$ such that
	\[
		\forall g,g'\in G \;:\; f(g\cdot g') = f(g) * f(g').
	\]
\end{definition}

\begin{definition}
	A \emph{ring homomorphism} from $(R,+,\cdot)$ to $(S,\oplus,*)$ is a function $f:R\to S$ such that for all $r,r'\in R$,
	\begin{enumerate}[i.]
		\item $f(r+r')=f(r)\oplus f(r')$,
		\item $f(r\cdot r') = f(r)*f(r')$,
		\item $f(1_R)=1_S$.
	\end{enumerate}
\end{definition}

\begin{definition}
	A \emph{field homomorphism} is a ring homomorphism between fields.
\end{definition}

\begin{definition}
	A \emph{monotone map} of posets from $(A,\leq)$ to $(B,\preccurlyeq)$ is a function $f:A\to B$ such that
	\[
		\forall a,a'\in A : a\leq a' \implies f(a) \preccurlyeq f(a').
	\]
	An \emph{order embedding} is an injective monotone map, and an \emph{order isomorphism} is a bijective monotone map.
\end{definition}

\begin{definition}
	A \emph{linear map} of $k$-vector spaces from $U$ to $V$ is a function $f:U\to V$ such that
		\begin{enumerate}[i.]
			\item $f(u+u') = f(u) + f(u')$ for all $u,u'\in U$, and
			\item $f(su) = sf(u)$ for all $u\in U$ and $s\in k$.
		\end{enumerate}
\end{definition}


\section*{Metric spaces}

\begin{definition}
	A \emph{metric} on $X$ is a function
	\[
		d:X\times X \to \R_{\geq0}
	\]
	satisfying
	\begin{enumerate}[i.]
		\item $d(x,y)=0$ if and only if $x=y$,
		\item $d(x,y) = d(y,x)$,
		\item $d(x,z) \leq d(x,y) + d(y,z)$ \hspace{.7cm}(\emph{triangle inequality}).
	\end{enumerate}
	The pair $(X,d)$ is called a \emph{metric space}.
\end{definition}

\begin{definition}
	A \emph{norm} on a vector space $V$ is a function
	\[
		\|\cdot\| : V\to\R_{\geq0}
	\]
	such that
	\begin{enumerate}[i.]
		\item $\|v\| = 0$ if and only if $v=0$,
		\item $\|sv\| = |s| \,\|v\|$,
		\item $\|u+v\| \leq \|u\| + \|v\|$ \hspace{.7cm}(\emph{triangle inequality}).
	\end{enumerate}
	The pair $(V,\|\cdot\|)$ is called a \emph{normed vector space}.
\end{definition}

\begin{definition}
	Let $(x_i)_i$ be a sequence in $(X,d)$ and fix $x\in X$. We say that $(x_i)_i$ \emph{converges} to $x$ if
	\[
		\forall \epsilon>0 : \exists N\in \N : \forall n\geq N : d(x_n,x) \leq \epsilon.
	\]
	In this case, we write $x_i\to x$ or $\displaystyle\lim_{i\to\infty}x_i = x$ and we say that $x$ is the \emph{limit} of $(x_i)_i$.
\end{definition}

\begin{definition}
	If the sequence $(x_i)_i$ does not converge to any point $x\in X$, then $(x_i)_i$ is said to \emph{diverge}.
\end{definition}


\section*{Continuous functions}

\begin{definition}
	A function $f:X\to Y$ is \emph{continuous at $x\in X$} when
	\[
		\forall \epsilon>0 : \exists \delta>0 : \forall y\in X : d_X(x,y) <\delta \implies d_Y(f(x),f(y)) <\epsilon.
	\]
\end{definition}

\begin{definition}
	We say that $f:X\to Y$ is \emph{continuous} if it is continuous at every $x\in X$.
\end{definition}


\section*{Cardinality}

\begin{definition}
	Two sets $A$ and $B$ are \emph{equivalent} (or in \emph{one-to-one correspondence}) if there exists a bijection from $A$ to $B$. In this case, we write $A\approx B$.
\end{definition}

\begin{definition}
	The \emph{cardinality} of a finite set $A=\{a_1,\ldots, a_k\}$ is the number $k\in\N$ of elements in $A$.
\end{definition}

\begin{definition}
	We write
	\begin{align*}
		|A| = |B|	&\text{ when } \exists \text{ bijection }f:A\xrightarrow{\sim} B,		\\[2pt]
		|A| \leq |B|	&\text{ when } \exists \text{ injection }f:A\hookrightarrow B,			\\[2pt]
		|A| < |B|	&\text{ when } |A|\leq |B| \text{ and } |A|\neq|B|.
	\end{align*}
\end{definition}









\end{document}





































