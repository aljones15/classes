\documentclass{article}
\usepackage{amsmath,amssymb,amsfonts,amsthm}
\usepackage[shortlabels]{enumitem}
\usepackage{fullpage}

% mathematical environments - see https://en.wikibooks.org/wiki/LaTeX/Theorems

\theoremstyle{plain}
\newtheorem{lemma}{Lemma}
\newtheorem{proposition}{Proposition}
\newtheorem{theorem}{Theorem}
\newtheorem{corollary}{Corollary}

\theoremstyle{definition}
\newtheorem{definition}{Definition}

\theoremstyle{remark}
\newtheorem{remark}{Remark}
\newtheorem{example}{Example}

% commands

\newcommand{\N}{\mathbb{N}}
\newcommand{\Z}{\mathbb{Z}}
\newcommand{\Q}{\mathbb{Q}}
\newcommand{\R}{\mathbb{R}}

\renewcommand{\emptyset}{\varnothing}
\renewcommand{\subset}{\subseteq}
\renewcommand{\supset}{\supseteq}
\renewcommand{\d}{\mathrm{d}}



\title{Parsing Binomials \& Multinomials in Probability}
\author{Andrew Jones}\date{}

\begin{document}\maketitle


% INTRODUCTION

\section*{Introduction}

Note that
\begin{enumerate}[i.,noitemsep]
	\item first point
	\item second point
	\item third point
\end{enumerate}



% FIRST SECTION

\section{Binomial Theorem}
The binomial theorem expresses the expansion of two monomial terms such as $(x + y)^2 = x^2 + 2xy + y^2$. However, in probability the binomial theorem can express the total probability of two independent events.
\begin{example}
  Let an unfair coin be flipped twice with $P(Tails) = 0.3$ and $P(Heads) = 0.7$ \\
  We know the probability must sum to 1. In two flips then, $(T + H)^2 = T^2 + 2TH + H^2$. This aligns with the outcomes of $TT$, $TH$, $HT$, and $HH$ for 2 flips. Substituting in the probabilities we have $0.3^2 + 2*0.3*0.7 + 0.7^2 = 1$.
\end{example}
To compute the expansion of two monomials the theorem uses:
% defintion 1
\begin{definition}[Factorial n!]
  Count everyway to permute a set of n distinct objects
  \begin{align*}
    n! = {\displaystyle\prod_{i=1}^{n}i}
  \end{align*}
  where $0! = 1$ and $n \geq 0$.
\end{definition}
Building on factorials, the theorem uses the
% definition 2
\begin{definition}[Binomial Coefficient]
  Count everyway to combine a set of n objects of k size
  \begin{align*}
    \binom{n}{k} = \frac{n!}{k!(n-k)!}
  \end{align*}
\end{definition}
Note that the binomial coeffecient can be used to determine the number of combinations of a particular size the binomial theorem produces.
\begin{example}
  3 flips of a coin yields:
  \begin{align*}
  &(T + H)^3 = TTT + TTH + THT + HTT + HHT+ HTH + THH + HHH \\
  &= T^3 + 3T^2H + 3H^2T + H^3 \\
  &\binom{3}{3} = 1\ \text{hence}\ T^3 = TTT\ \text{or}\ H^3 = HHH \\
  &\binom{3}{2} = 3\ \text{hence}\ 3T^2H = TTH + THT + HTT\ \text{or}\ 3H^2T = HHT+HTH+THH
  \end{align*}
\end{example}
Many combinations can be simplified by the use of Pascal's identity:
% proof 1
\begin{proof}[Pascal's Identity]
  \begin{align*}
    \binom{n - 1}{k} + \binom{n - 1}{k - 1} &= \frac{(n-1)!}{k!(n - 1 - k)!} + \frac{(n - 1)!}{(k-1)!(n-k)!} \\
    &= (n - 1)![\frac{n-k}{k!(n-k)!} + \frac{k}{k(n-k)!}] \\
    &= (n - 1)!\frac{n}{k!(n-k)!} \\
    &= \frac{n!}{k!(n - k)!} \\
    &= \binom{n}{k}
  \end{align*}
\end{proof}
% add proof for Binomial Theorem here
% proof 2
\begin{proof}[Binomial Theorem]

\end{proof}

% SECOND SECTION

\section{Multinomial Theorem}

\begin{lemma}
	We have
	\[
		\int_0^\pi \sin(3x)\,\d x = \frac{2}{3}.
	\]
\end{lemma}

% proof 3
\begin{proof}
	A direct computation yields
	\begin{align*}
		\int_0^\pi \sin(3x)\,\d x
			&=	\frac{1}{3}\int_0^{3\pi} \sin u \,\d u,		\hspace{1cm} u=3x,		\\
			&=	\frac{1}{3} \big[-\cos u\big]_0^{3\pi}						\\
			&=	\frac{1}{3} \big[1-(-1)\big]						\\
			&=	\frac{2}{3}.
	\end{align*}
\end{proof}

\begin{remark}
	This is interesting since\textellipsis
\end{remark}

\section{Possible Outcomes to Equations}



\end{document}
