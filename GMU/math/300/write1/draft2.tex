\documentclass{article}
\usepackage{amsmath,amssymb,amsfonts,amsthm}
\usepackage[shortlabels]{enumitem}
\usepackage{fullpage}

% mathematical environments - see https://en.wikibooks.org/wiki/LaTeX/Theorems

\theoremstyle{plain}
\newtheorem{lemma}{Lemma}
\newtheorem{proposition}{Proposition}
\newtheorem{theorem}{Theorem}
\newtheorem{corollary}{Corollary}

\theoremstyle{definition}
\newtheorem{definition}{Definition}

\theoremstyle{remark}
\newtheorem{remark}{Remark}
\newtheorem{example}{Example}

% commands

\newcommand{\N}{\mathbb{N}}
\newcommand{\Z}{\mathbb{Z}}
\newcommand{\Q}{\mathbb{Q}}
\newcommand{\R}{\mathbb{R}}

\renewcommand{\emptyset}{\varnothing}
\renewcommand{\subset}{\subseteq}
\renewcommand{\supset}{\supseteq}
\renewcommand{\d}{\mathrm{d}}



\title{Parsing Binomials \& Multinomials in Probability}
\author{Andrew Jones}\date{}

\begin{document}\maketitle


% INTRODUCTION

\section*{Introduction}

The application of the Binomial and Multinomial theorems in probability can often lack clarity. However, the relations between polynomials and probability can provide insight into how models work in probability. This article aims to clarify the connection between expansions of multinomials and binomials, combinations, total probability, and probability mass functions.

% FIRST SECTION

\section{Binomial Theorem}
The Binomial Theorem expresses the expansion of two monomial terms $x$ and $y$ such that $(x + y)^2 = x^2 + 2xy + y^2$. The later is useful in algebra and other fields of mathematics, but how a binomial expansion relates to probability is not intuitvely obvious. The solution is to treat $x$ and $y$ as the two possible outcomes of an independent event. The resulting binomial expansion can express the sample space or total probability of all combinations of the two outcomes for multiple independent events.
\begin{example}
  Let an unfair coin be flipped twice with $P(Tails) = 0.3$ and $P(Heads) = 0.7$ \\
  We know the probability must sum to 1. In two flips then, $(T + H)^2 = T^2 + 2TH + H^2$. This aligns with the outcomes of $T^2 = TT$, $2TH = TH + HT$, and $H^2 = HH$ for 2 flips. Additionally, substituting in the probabilities we have $0.3^2 + 2*0.3*0.7 + 0.7^2 = 1$.
\end{example}
\begin{theorem}
  The Binomial Theorem can express the sum of all possible outcomes of multiple independent events.
\end{theorem}
\begin{proof}
  Let $A$ and $B$ be the two outcomes of $n$ independent events with probability $A = \frac{1}{a}$, $B = (1 - \frac{1}{a})$, and $n = 2$. By the binomial theorem we have
  \begin{align*}
    &(\frac{1}{a} + (1-\frac{1}{a}))^2 \\
    &(\frac{1}{a})^2 + 2\frac{1}{a}(1-\frac{1}{a}) + (1-\frac{1}{a})^2 \\
    &\frac{1}{a^2} + \frac{2}{a}\frac{a-1}{a} + \frac{(a-1)^2}{a^2} \\
    &\frac{1}{a^2} + \frac{2(a-1)}{a^2} + \frac{a^2-2a+1}{a^2} \\
    &\frac{a^2+2-2+2a-2a}{a^2} \\
    &\frac{a^2}{a^2} = 1
  \end{align*}
\end{proof}
The uses of the Binomial Theorem are obvious especially in calculating large numbers of events. To compute the expansion of two monomials the theorem uses:
% defintion 1
\begin{definition}[Factorial n!]
  The factorial of a nonnegative integer $n$ is given by
  \begin{align*}
    n! = {\displaystyle\prod_{i=1}^{n}i}
  \end{align*}
  where $0! = 1$.
\end{definition}
Building on factorials, the theorem uses the
% definition 2
\begin{definition}[Binomial Coefficient]
  Count everyway to combine a set of $n$ objects of $k$ size
  \begin{align*}
    \binom{n}{k} = \frac{n!}{k!(n-k)!}
  \end{align*}
\end{definition}
Note that the binomial coeffecient can be used to determine the number of combinations of a particular size the binomial theorem produces.
\begin{example}
  3 flips of a coin yields:
  \begin{align*}
  &(T + H)^3 = TTT + TTH + THT + HTT + HHT+ HTH + THH + HHH \\
  &= T^3 + 3T^2H + 3H^2T + H^3 \\
  &\binom{3}{3} = 1\ \text{hence}\ T^3 = TTT\ \text{or}\ H^3 = HHH \\
  &\binom{3}{2} = 3\ \text{hence}\ 3T^2H = TTH + THT + HTT\ \text{or}\ 3H^2T = HHT+HTH+THH \\
  \end{align*}
\end{example}
\begin{remark}
  The total number of combinations of the binomial is $2^n$. \\
  $2^3 = T^3 + 3T^2H + 3H^2T + H^3 = 1 + 3 + 3 + 1 = 8$
\end{remark}
Many combinations can be simplified by the use of Pascal's identity:
% proof 1
\begin{proof}[Pascal's Identity]
  \begin{align*}
    \binom{n - 1}{k} + \binom{n - 1}{k - 1} &= \frac{(n-1)!}{k!(n - 1 - k)!} + \frac{(n - 1)!}{(k-1)!(n-k)!} \\
    &= (n - 1)![\frac{n-k}{k!(n-k)!} + \frac{k}{k(n-k)!}] \\
    &= (n - 1)!\frac{n}{k!(n-k)!} \\
    &= \frac{n!}{k!(n - k)!} \\
    &= \binom{n}{k}
  \end{align*}
\end{proof}
% add proof for Binomial Theorem here
% proof 2
\begin{proof}[Binomial Theorem]
  Assume that $(a + b)^n = \sum_{k=0}^{n} \binom{n}{k} a^kb^{n-k}$ and by the the definition of the binomial coeffecient $n \geq 0$. For the case $(n = 0) \Rightarrow (a + b)^0 = 1$. For the case $n \ge 0$.
  \begin{align*}
    (a + b)^{n+1} &= (a + b)(a + b)^n = (a+b)\sum_{k=0}^{n}\binom{n}{k}a^kb^{n-k} \\
                  &= \sum^{n}_{k=0}\binom{n}{k}a^{k+1}n{n-k} + \sum^{n}_{k=o}\binom{n}{k}a^{k}b^{n-k+1} \\
    m &= k + 1 \\
                  &= \sum^{n+1}_{m=1}\binom{n}{m-1}a^{m}b^{n-m+1}+\sum^{n}_{k=0}\binom{n}{k}a^{k}b^{n-k+1} \\
                  &= b^{n+1}+\sum^{n}_{k=1}[\binom{n}{k} + \binom{n}{k-1}]a^kb^{n-k+1}+a^{n+1}\\
                  &= b^{n+1}+\sum^{n}_{k=1}\binom{n+1}{k}a^{k}b^{n-k+1}+a^{n+1} \\
                  &= \sum^{n+1}_{k}\binom{n+1}{k}a^{k}b^{n+1-k}
  \end{align*}
\end{proof}

% SECOND SECTION

\section{Multinomial Theorem}
While the binomial theorem applies to two independent events, the multinomial theorem generalizes to any number of groups or events. The theorem uses the multinomial coeffecient.
% definition 3
\begin{definition}[Multinomial Coeffecient]
  \begin{align*}
    \binom{N}{n_1..n_r} = \frac{N!}{n_1!..n_r!}
  \end{align*}
  Where $n_1$ to $n_r$ are different group sizes.
\end{definition}

\begin{example}
  For 13 items we want to know how many combinations of 5, 5, and 3 can be made
  \begin{align*}
    \binom{13}{5,5,3} &= \binom{13}{5}\binom{8}{5}\binom{3}{3} \\
                      &= \frac{13!}{5!(13-5)!}\frac{8!}{5!(8-5)!}\frac{3!}{3!(3-3)!} \\
                      &=\frac{13!}{5!5!3!}
  \end{align*}
\end{example}
We now combine the multinomial coeffecients with monomials raised to a power to get:
\begin{definition}[multinomial theorem]
  \begin{align*}
    (x_1+...+x_r)^n = \sum_{(n1,...,nr)}\binom{n}{n1,...,nr}x_1^{n1}...x_r^{nr} \\
    \text{where}\ n1 + ... + nr = n
  \end{align*}
\end{definition}
% proof
\begin{proof}[Multinomial Theorem]
  Fix $r = 1$ and observe that $(x_1)^n=\sum_{(n1 = 1)}\binom{n}{n1 = 1}x_1^{n1 = 1} = nx_1$ \\
  Fix $m = r + 1$ and $(x_r + x_{r+1})^n = \sum_{(r, ..., r+1)}\binom{n}{x_1,...,x_{r+1}}x_1^rx_{r+1}^{r+1}$  and observe that this is provable using the binomial theorem.
\end{proof}


\end{document}
