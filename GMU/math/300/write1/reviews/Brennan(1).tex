\documentclass{article}
\usepackage[dvipsnames]{xcolor}				% see  https://en.wikibooks.org/wiki/LaTeX/Colors
\usepackage[backgroundcolor=SkyBlue!30]{todonotes}	% see  https://tug.ctan.org/macros/latex/contrib/todonotes/todonotes.pdf
\usepackage{graphicx}
\usepackage{amsmath, amsfonts, amssymb, amsthm, fullpage, wasysym, dsfont}
\usepackage[shortlabels]{enumitem}
\usepackage[fontsize=10]{scrextend}
\usepackage{hyperref}

\theoremstyle{plain}
\newtheorem{lemma}{Lemma}
\newtheorem{proposition}{Proposition}
\newtheorem{theorem}{Theorem}
\newtheorem{corollary}{Corollary}

\theoremstyle{definition}
\newtheorem{definition}{Definition}

\theoremstyle{remark}
\newtheorem{remark}{Remark}
\newtheorem{example}{Example}

% commands

\newcommand{\N}{\mathbb{N}}
\newcommand{\Z}{\mathbb{Z}}
\newcommand{\Q}{\mathbb{Q}}
\newcommand{\R}{\mathbb{R}}

\renewcommand{\emptyset}{\varnothing}
\renewcommand{\subset}{\subseteq}
\renewcommand{\supset}{\supseteq}
\renewcommand{\d}{\mathrm{d}}

\title{Introduction to Calculus}
\author{Brennan Williams}
\date{}


\begin{document}

\maketitle

\section{Introduction to Derivatives}
The derivative of a function describes the rate at which that function is changing. \;Specifically, for a given function, the derivative is the change in the dependent variable with respect to an infinitesimal change in the independent variable.
\todo{Granted this is an advanced math course, but maybe start with a definition of a limit?}
\begin{definition}
    The \textit{derivative} of a function $f(x)$ with respect to $x$ is given by
    \begin{align*}
        f'(x)=\lim\limits_{h\to0} \frac{f(x+h)-f(x)}{(x+h)-x}=\lim\limits_{h\to0} \frac{f(x+h)-f(x)}{h} 
    \end{align*} 
\end{definition}

\begin{example}
    If $f(x)=2x$, then $f'(x)=\lim\limits_{h\to0}\frac{2(x+h)-2x}{h}=\lim\limits_{h\to0}\frac{2x+2h-2x}{h}=\lim\limits_{h\to0}\frac{2h}{h}=2$.
\end{example}

\textit{Remark}. \;For all $a\in$\;Dom\;($2x$), $f'(a)=2$.\\

\begin{example}
    If $f(x)=x^2$, then $f'(x)=\lim\limits_{h\to0}\frac{(x+h)^2-x^2}{h}=\lim\limits_{h\to0}\frac{x^2
    +2xh+h^2-x^2}{h}=\lim\limits_{h\to0}\frac{2xh+h^2}{h}=\lim\limits_{h\to0}(2x+h)=2x$.
\end{example}

\textit{Remark}. \;For all $a\in$\;Dom\;($x^2$), $f'(a)=2a$.

\section{The Meaning of Differentiability}
\noindent The derivative of a function may not exist at every point within an interval. \;When the derivative of a function does not exist at a given point, we say that the function is \textit{undifferentiable} at that point. \;Conversely, when the derivative of a function exists at every point within a specified interval, we say that the function is \textit{differentiable} over that interval.

\begin{definition}
    A function $f(x)$ is \textit{undifferentiable} at $x=a$ if $f'(a)$ does not exist.
\end{definition}

\textit{Remark}.
    In general, $f'(a)$ does not exist when $f(x)$ is discontinuous at $x=a$.\todo{Really impressed with your work, but maybe a defintion or example of discontinuous would help. I don't think the assumption the reader would know what discontinuous is at this level is unreasonable though.}

\begin{definition}
    A function $f(x)$ is  \textit{differentiable} on $[a,b]$ if $f'(x)$ exists at every $x\in[a,b]$.
\end{definition}


\section{Derivative Shortcuts}
\begin{theorem}
    The derivative of $f(x)=c$ with respect to $x$, where $c\in\R$, is given by $f'(x)=0$.\todo{Needs more context such as $c$ is an arbitrary constant}
\end{theorem} 

\begin{proof}
    By the definition of a derivative, we have
    	\begin{align*}
		f'(x)
			&=	\lim\limits_{h\to0}\frac{c-c}{h}\\
			&=	\lim\limits_{h\to0}\frac{0}{h}\\
			&=	0
	\end{align*}
\end{proof}

\begin{example}
    If $f(x)=2$, then $f'(x)=\lim\limits_{h\to0}\frac{2-2}{h}=\lim\limits_{h\to0}\frac{0}{h}=0$.
\end{example}

\begin{theorem}
    The derivative of $f(x)=x^a$ with respect to $x$, where $a\in\N$, is given by $f'(x)=ax^{a-1}$.
\end{theorem}
   
  \begin{proof}
	By the definition of a \textit{derivative}, we have
	\begin{align*}
		f'(x)
			&=	\lim\limits_{h\to0} \frac{(x+h)^a-x^a}{h}\\
			&=	\lim\limits_{h\to0} \frac{\sum^{a}_{i=0}{a\choose r}x^{a-r}h^r-x^a}{h}\\
			&=	\lim\limits_{h\to0} \frac{{a\choose 0} x^a+{a\choose 1}x^{a-1}h+{a\choose 2}x^{a-2}h^2+\;.\;.\;.\;+{a\choose a}h^a-x^a}{h}\\
			&=	\lim\limits_{h\to0} \frac{{a\choose 1}x^{a-1}h+{a\choose 2}x^{a-2}h^2+\;.\;.\;.\;+{a\choose a}h^a}{h}\\
			&=	\lim\limits_{h\to0} {a\choose 1}x^{a-1}+\lim\limits_{h\to0}{a\choose 2}x^{a-2}h +\;.\;.\;.\;+\lim\limits_{h\to0}{a\choose a}h^{a-1}\\
			&=	{a\choose 1}x^{a-1}\\
            &=ax^{a-1}
	\end{align*}
\end{proof}

\begin{example}
    If $f(x)=x^3$, then $f'(x)=3x^2$.
\end{example}

\begin{example}
    If $f(x)=2x^4$, then $f'(x)=8x^3$
\end{example}
    
\section{Derivatives of Exponential Functions}

\begin{theorem}
    The derivative of $f(x)=a^x$, where $a\in\R$, is $f'(x)=a^x\lim\limits_{h\to0} \frac{a^h-1}{h}$.
\end{theorem}
   
    \begin{proof}
	By the definition of a derivative, we have
	\begin{align*}
		f'(x)
			&=	\lim\limits_{h\to0} \frac{a^{x+h}-a^x}{(x+h)-x}\\
			&=	\lim\limits_{h\to0} \frac{a^x(a^h-1)}{h}\\
			&=	a^x\lim\limits_{h\to0} \frac{a^h-1}{h}
	\end{align*}
    \end{proof}

\textit{Remark}.
The $\lim\limits_{h\to0} \frac{a^h-1}{h}$ is a constant which depends on the value of $a$.

\begin{example}
    If $f(x)=1.5^x$, then $f'(x)=1.5^x\lim\limits_{h\to0} \frac{1.5^h-1}{h}\approx0.405\times1.5^x$.
\end{example}

\textit{Remark}.
    In this case, the $\lim\limits_{h\to0} \frac{1.5^h-1}{h}\approx0.405$.


\noindent Suppose there is an $a\in\R$ such that $\lim\limits_{h\to0} \frac{a^h-1}{h}=1$. \;Then, the derivative of $f(x)=a^x$ is $f'(x)=a^x$. \;The number which satisfies this supposition is Euler's constant.


\begin{definition}
    Euler's constant, denoted by $e$, is the real number such that $\lim\limits_{h\to0} \frac{e^h-1}{h}=1$.
\end{definition}

\textit{Remark}.
If $f(x)=e^x$, then $f'(x)=e^x\lim\limits_{h\to0} \frac{e^h-1}{h}=e^x$.

\newpage
\section*{Bibliography}
Sections 1-3\\
\indent Calculus Early Transcendentals: Differential \& Multi-Variable Calculus for Social Sciences. 
4 Derivatives.  \url{https://www.sfu.ca/math-coursenotes/Math\%20157\%20Course\%20Notes/sec_TheDerivativeFunction.html}\\\\

\noindent Section 4\\
 \indent Blackpenredpen. 
 Why is the derivative of $e^x$ equal to $e^x$? \;\url{https://www.youtube.com/watch?v=oBlHiX6vrQY&t=412s}.

 

\end{document}

