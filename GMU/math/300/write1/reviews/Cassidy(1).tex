\documentclass[10pt]{article}
\usepackage[dvipsnames]{xcolor}				% see  https://en.wikibooks.org/wiki/LaTeX/Colors
\usepackage[backgroundcolor=SkyBlue!30]{todonotes}	% see  https://tug.ctan.org/macros/latex/contrib/todonotes/todonotes.pdf

\usepackage{amsmath, amssymb, amsthm, fullpage}

\newtheorem{theorem}{Theorem}
\newtheorem{definition}{Definition}

\title{Written Assignment \#1: \\{\Large An Introduction to the Slope-Intercept Form in Elementary Algebra}}

\author{Cassidy Rivers}
\date{March 4th 2025}

\begin{document}

\maketitle

\section{\textbf{Introduction}}
One of the fundamental concepts in \textbf{elementary algebra} is the equation of a line, which can be represented in various forms. Among these, the \textbf{slope-intercept form} is one of the most widely used due to its simplicity and direct relationship to the graph of a line. This paper introduces the slope-intercept form, provides essential definitions, and proves key properties related to graphing linear equations. \todo{This is simple, concise, and perfect. Good job.}

\section{\textbf{Definitions}}
We begin with the essential definitions necessary to understand the slope-intercept form of a line.

\begin{definition}
The slope-intercept form of a \textbf{linear equation} is given by:
\[
y = mx + b
\]
where:
\begin{itemize}
    \item \( m \) represents the \textbf{slope} of the line, which determines its steepness.
        \item \( b \) represents the \textbf{y-intercept}, which is the point where the line crosses the \( y \)-axis.
\end{itemize}
\end{definition}

\begin{definition}
The slope of a line passing through two points \( (x_1, y_1) \) and \( (x_2, y_2) \) is given by:
\[
m = \frac{y_2 - y_1}{x_2 - x_1}, \quad \text{for } x_1 \neq x_2.
\]
The slope measures the rate of change of \( y \) with respect to \( x \).
\end{definition}

\begin{definition}
A line is said to be:
\begin{itemize}
    \item \textbf{Parallel} to another line if they have the same slope, i.e., \( m_1 = m_2 \).
    \item \textbf{Perpendicular} to another line if the product of their slopes is \( -1 \), i.e., \( m_1 \cdot m_2 = -1 \). \todo{This seems like 2 definitions in one. Maybe break them apart with examples?}
\end{itemize}
\end{definition}

\section{\textbf{Theorems and Proofs}}
We now prove some key properties of the slope-intercept form.

\begin{theorem}
Given a linear equation \( Ax + By = C \), it can always be rewritten in the slope-intercept form \( y = mx + b \), provided \( B \neq 0 \).
\end{theorem}

\begin{proof}
Starting from the general linear equation:
\[
Ax + By = C.
\]
Solving for \( y \), we subtract \( Ax \) from both sides:
\[
By = -Ax + C.
\]
Dividing by \( B \) (assuming \( B \neq 0 \)):
\[
y = -\frac{A}{B}x + \frac{C}{B}.
\]
Thus, the equation is now in the form of slope intercept, where \( m = -\frac{A}{B} \) and \( b = \frac{C}{B} \).
\end{proof}

\begin{theorem}
The slope of a line is constant, which means that for any two points \( (x_1, y_1) \) and \( (x_2, y_2) \) on the same line, the slope calculation always yields the same value.
\end{theorem}

\begin{proof}
For a linear equation \( y = mx + b \), consider two arbitrary points on the line: \( (x_1, y_1) \) and \( (x_2, y_2) \). By definition, we have:
\[
y_1 = m x_1 + b, \quad y_2 = m x_2 + b.
\]
Computing the slope:
\[
m = \frac{y_2 - y_1}{x_2 - x_1} = \frac{(m x_2 + b) - (m x_1 + b)}{x_2 - x_1}.
\]

Simplifying:
\[
m = \frac{m x_2 - m x_1}{x_2 - x_1} = m \frac{x_2 - x_1}{x_2 - x_1} = m.
\]
Thus, the slope remains the same for any two points on the line.
\end{proof}

\begin{theorem}
The equation of a line passing through a point \( (x_0, y_0) \) with a given slope \( m \) is given by:
\[
y - y_0 = m(x - x_0).
\]
\end{theorem}

\begin{proof}
By the definition of the slope, we have:
\[
m = \frac{y - y_0}{x - x_0}.
\]
Rearranging:
\[
y - y_0 = m(x - x_0).
\]
This equation is known as the point-slope form of a line, which can be rewritten in slope intercept form if needed.\todo{A little more context for these theorems and maybe some examples that use them would be nice.}
\end{proof} 

\section{\textbf{Graphing the Slope-Intercept Form}}
The slope-intercept form of a line provides a straightforward way to graph linear equations The steps are as follows:

\begin{enumerate}
    \item Identify the slope \( m \) and the y-intercept \( b \).
        \item Plot the y intercept \( (0, b) \).
            \item Use the slope \( m \) to determine another point by counting the rise and run.
                \item Draw a straight line through the points. \todo{I like where this is going, but it needs work and maybe a graph.}
\end{enumerate}

\section{\textbf{Conclusion}}
The slope-intercept form is an essential tool in elementary algebra, offering an intuitive method for graphing linear equations. Overall this paper covered the fundamental definitions, key theorems, and properties of the slope-intercept form, along with an explanation of how to graph equations in this form.

\end{document}
