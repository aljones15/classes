\documentclass{article}
\usepackage{amsmath,amssymb,amsthm}
\usepackage{fullpage}
\usepackage{comment}

\newcommand{\R}{\mathbb{R}}
\newcommand{\Z}{\mathbb{Z}}
\newcommand{\N}{\mathbb{N}}

\renewcommand{\emptyset}{\varnothing}
\renewcommand{\subset}{\subseteq}



\title{Worksheet 3 Answer Key}
\author{}\date{}



\begin{document}
\maketitle

\noindent
We will prove 3, 4, 5, 6, 10, and 11, and disprove 1, 2, and 9.

\begin{enumerate}
	\item Let $A=C=\{0\}$ and $B=\{1\}$. We have $\{0\}\cap\{1\}=\emptyset$ while $\{0\}\cap\{0\}\neq\emptyset$.
	\item Let $A=C=\{0\}$ and $B=\{1\}$. We have $\{0\}\not\subset\{1\}$ and $\{1\}\not\subset\{0\}$, while $\{0\}\subset\{0\}$.
	\item Suppose that $A\subset\emptyset$. From $\emptyset\subset A$, we conclude that $A=\emptyset$.
	\item Fix $x\in A\cap B$. By the definition of intersection, we have $x\in A$ and $x\in B$. From the inclusion $A\subset C$, it follows that $x\in C$.
	\item Let $a,a'\in A$ and suppose that $g(f(a)) = g(f(a'))$. By the injectivity of $g$ we obtain $f(a)=f(a')$, and by the injectivity of $f$ we conclude that $a=a'$.
	\item Fix $c\in C$. The surjectivity of $g$ implies the existence of a $b\in B$ with $g(b)=c$, while that of $f$ yields an $a\in A$ with $f(a)=b$. Observe that, $g(f(a))=g(b)=c$.
	\item Consider $f:\N\to\N$ defined by
		\[
			f(k) = k+1.
		\]
	\item Let $f:\N\to\N$ be given by
		\[
			f(k) =
				\begin{cases}
					k-1	&\text{ if }k\geq 1	\\
					0	&\text{ if }k=0.
				\end{cases}
		\]
	\item Let $f:\{0\}\hookrightarrow\N$ be the inclusion map, and let $g:\N\to\{0\}$ be the constant map with value $0$. Then
		\begin{align*}
			g\circ f:	\{0\}	&\to		\{0\}		\\
					0\;	&\mapsto	\;0
		\end{align*}
		is a bijection, while neither $f$ nor $g$ is a bijection.
	\item Suppose for a contradiction that $f:A\to A$ is not injective. It follows that the size of the image of $f$ is strictly less than the size of the domain $A$. Consequently, the image of $f$ is not equal to $A$. This contradicts the surjectivity of $f$.
	\item Let $x,x'\in A$ with $f(x)=f(x')$. By applying $f$ to both sides of the preceding equality, we have
		\[
			x = f\circ f(x) = f\circ f(x') = x'
		\]
		whence $f$ is injective. Now fix $y\in A$. From the identity $f(f(y)) = y$ we conclude that $f$ is surjective.
\end{enumerate}



\end{document}







