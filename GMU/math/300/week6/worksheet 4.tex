\documentclass{article}
\usepackage{../../notes/math}

\title{Worksheet 4}
\author{}\date{}



\begin{document}\maketitle
\thispagestyle{empty}

\vspace{0.7cm}\noindent
Name: \hspace{0.1cm}\underline{\hspace{7cm}}\\[5pt]
Due by midnight of \textbf{Monday}, Mar.\ 3, on Gradescope.



\vspace{.8cm}\noindent
Let $R$ be a relation from $A$ to $B$, let $S$ be a relation from $B$ to $C$, and let $T$ be a relation from $C$ to $D$. Prove the following statements.
\begin{enumerate}
	\item $I_A\circ R = R$
	\item $R\circ I_B = R$
	\item $(R^{-1})^{-1}=R$
	\item $(S\circ R)^{-1} = R^{-1}\circ S^{-1}$
	\item $(T\circ S)\circ R = T\circ (S\circ R)$
	\item $\mathrm{Dom}\,R = \mathrm{Rng}\,R^{-1}$
	\item $\mathrm{Rng}\,R = \mathrm{Dom}\,R^{-1}$
\end{enumerate}

\vspace{8pt}\noindent
For Question 8--10, suppose that $A=B=C$.

\begin{enumerate}\setcounter{enumi}{7}
	\item If $R$ and $S$ are equivalence relations, then it is not necessarily true that $S\circ R$ is an equivalence relation.
	\item If $R$ is a partial order, then $R\circ R$ is a partial order.
	\item If $R$ and $S$ are partial orders, then it is not generally true that $S\circ R$ is a partial order.
\end{enumerate}

\vspace{8pt}\noindent
\textbf{Bonus questions.}

\vspace{.8pt}\noindent
Give an example of two relations $R$ and $S$ on a set $A$ such that
\begin{enumerate}\setcounter{enumi}{10}
	\item $R\circ S\neq S\circ R$.
	\item $S\circ R$ is an equivalence relation, but neither $R$ nor $S$ is an equivalence relation.
\end{enumerate}






\end{document}





































