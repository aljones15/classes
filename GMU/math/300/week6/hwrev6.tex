\documentclass{article}
  \usepackage{amsfonts}
  \usepackage{amsthm}
  \usepackage{mathtools}
\begin{document}
    MATH-300 \hfill Andrew Jones
  \begin{center}
  {\huge Worksheet 4}
  \end{center}
  Let $R$ be a relation from $A$ to $B$, let $S$ be a relation from $B$ to $C$, and let $T$ be a relation from $C$ to $D$. \\ Prove the following statements. \\
  \begin{enumerate}
    % 1 correct
    \item $I_A \circ R = R$
      \begin{proof}
        Let $a\in A$ and $b\in B$ and observe:
        \begin{align*}
          a(I_a \circ R)b &\iff \exists a^`\in A: a = a^` \wedge a^`Rb \\
          &\iff aRb
        \end{align*}
      \end{proof}
    % 2 correct
    \item $R \circ I_A = R$
      \begin{proof}
        Let $a\in A$ and $b\in B$ and observe:
        \begin{align*}
          a(R \circ I_a)b &\iff \exists b^{'}\in B: b = b^{'} \wedge aRb^{'} \\
          &\iff aRb
        \end{align*}
      \end{proof}
    % 3 correct
    \item $(R^{-1})^{-1} = R$
      \begin{proof}
        Fix $a\in A$ and $b\in B$:
        \begin{align*}
          a(R^{-1})^{-1}b &\iff bR^{-1}a \\
          &\iff b(R^{-1})^{-1}a
        \end{align*}
      \end{proof}
      % 4 correction needed here
      % no comment from prof
    \item $(S \circ R)^{-1} = R^{-1}\circ S^{-1}$
      \begin{proof}
        Suppose $(c,a)\in (S \circ R)^{-1}$. Then by implication: $a (S \circ R)^{-1}$. Hence, there exists a $b \in B$ such that $bSc$ and $aRb$ and $cS^{-1}b$ and $bR^{-1}a$. Therefore $(c,a)\in (R^{-1} \circ S^{-1})$ and $(R^{1} \circ S^{-1}) \subseteq (S \circ R)^{-1}$. The converse implication is obtained by retracing the given steps.
      \end{proof}
      % 5 correction needed here
      % no comment from prof
    \item $(T \circ S) \circ R = T \circ (S \circ R)$
      \begin{proof}
        Assume $(a,d)\in (T \circ S) \circ R$. It follows that $b\in B$ such that $aRb$ and $b(T \circ S)d$. Hence there is a $c\in C$ such that $bSc$ and $cTd$. This implies $a(S \circ R)c$, hence $aT\circ (S \circ R)d$. So we can conclude $T \circ (S \circ R) \subseteq (T \circ S) \circ R$. The converse implication is similar.
      \end{proof}
      % 6 correction needed here
      % It is unclear how you are using the definition of inverse or
      % that of the domain in order to arrive at your conclusions.
      % Additionally, all relations are invertible.
    \item $Dom\,R = Rng\,R^{-1}$
      \begin{proof} ($\subseteq$)
        Fix $a \in A$ and observe that $a \in Dom\,R$. There there must exist $b\in B$ such that $aRb$ and $bR^{-1}a$. Hence $a \in Rng\,R^{-1}$ and $Rng\,R^{-1} \subseteq Dom\,R$.
      \end{proof}
      \begin{proof} ($\supseteq$)
Fix $a \in A$ and observe $a \in Rng\,R^{-1}$. There must be $b\in B$ such that $bR^{1}a$ and $aRb$. Hence $a \in Dom\,R$ and $Dom\,R \subseteq Rng\,R^{-1}$.
      \end{proof}
      % 7 correction needed here
      % (Same comments as for Question 6.)
    \item $Rng\,R = Dom\,R^{-1}$
      \begin{proof} ($\supseteq$)
        Suppose $b\in Rng\,R$. This implies $a \in A$ such that $aRb$ and $bR^{1}a$. Hence by the invertibility of R, $b\in Dom\,R^{-1}$ and $Dom\,R^{-1} \subseteq Rng\,R$.
      \end{proof}
      \begin{proof} ($\subseteq$)
Fix $b\in Dom\,R^{-1}$. By implication we have $a \in A$ such that $bR^{-1}a$ and $aRb$. So it follows that $b \in Rng\,R$ and $Rng\,R \subseteq Dom\,R^{-1}$.
      \end{proof}
  \end{enumerate}
  For Question 8–10, suppose that $A = B = C$.
  \begin{enumerate} \setcounter{enumi}{7}
    % 8 reflexive, transitive, and symmetric
    % correct, no fix needed
    \item If $R$ and $S$ are equivalence relations, then $S \circ R$ is an equivalence relation.
      \begin{proof}
        Suppose $R$ is an equivalence relation from $A$ to $B$ and $S$ is an equivalence relation from $B$ to $C$ and $A = B = C$.
        \begin{align*}
          S \circ R &\iff \forall a\in A : aSa \wedge aRa \\
          &\iff \forall a,b,c\in A: (aSb \wedge bSc) => aSa \wedge (aRb \wedge bRc) => aRc \\
            &\iff \forall a,b\in A: (aSb \wedge bSa) \wedge (aRb \wedge bRa) 
        \end{align*}
      \end{proof}
    % 9 reflexive, transitive, antisymmetric
    % correction needed for 9
    \item If $R$ is a partial order, then $R \circ R$ is a partial order.
      \begin{proof}
        Fix $a,b,c\in A$:
        \begin{align*}
          aRc &\iff aRa \wedge bRb \wedge cRc \\
            &\iff (aRb \wedge bRc) \Rightarrow aRc \\
            &\iff (aRb \wedge bRa) \Rightarrow a = b \\
            &\iff (a(R \circ R)b \wedge b(R \circ R)a) \Rightarrow a = b \\
            &\iff a(R \circ R)a \wedge b(R \circ R)b \wedge c(R \circ R)c \\
            &\iff (a(R \circ R)b \wedge b(R \circ R)c) \Rightarrow a(R \circ R)c \\ 
            &\iff a(R \circ R)c
        \end{align*}
      \end{proof}
      % 10 reflexive, transitive, antisymmetric
      % correction needed for 10
    \item If $R$ and $S$ are partial orders, then it is not generally true that $S \circ R$ is a partial order.
      \begin{proof}
        Let $R = \le$ and $S = |$ therefore $a(S \circ R)c =\, a\le\, b | c$ where b is both less than a and a divisor of c. Fix $a = 5$ and $c = 3$. We have $5\le\, b|3$. Observe that there is no integer b that is a divisor of 3 and greater then or equal to 5. 
      \end{proof}
  \end{enumerate}
  \textbf{Bonus Questions}
  Give an example of two relations $R$ and $S$ on a set $A$ such that \\
  \begin{enumerate} \setcounter{enumi}{10}
    % 11
    \item $R \circ S \neq S \circ R$.
      \begin{proof}
        Suppose $R = \le$ and $S = |x|$. Fix $a = -9$ and $b = 5$. Observe that $-9 (R \circ S)5 \neq -9 (S \circ R) 5$.  
      \end{proof}
    % 12
    \item $S \circ R$ is an equivalence relation, but neither $R$ nor $S$ is an equivalence relation.
      \begin{proof}
        
      \end{proof}
   \end{enumerate}
\end{document}
