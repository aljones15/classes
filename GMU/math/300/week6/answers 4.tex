\documentclass{article}
\usepackage{amsmath,amssymb,amsthm}
\usepackage{fullpage}

\newcommand{\R}{\mathbb{R}}
\newcommand{\Z}{\mathbb{Z}}
\newcommand{\N}{\mathbb{N}}

\DeclareMathOperator{\Dom}{Dom}
\DeclareMathOperator{\Rng}{Rng}



\title{Worksheet 4 Answer Key}
\author{}\date{}



\begin{document}\maketitle

\begin{enumerate}
	\item Let $a\in A$ and $b\in B$ and observe that
		\begin{align*}
			a(I_A\circ R)b
				&\iff	\exists a'\in A : aI_Aa' \wedge a'Rb	\\
				&\iff	aRb.
		\end{align*}
	\item Fix $a\in A,b\in B$ and note that
		\begin{align*}
			a(R\circ I_B)b
				&\iff	\exists a'\in A : aRb' \wedge b'I_Bb	\\
				&\iff	aRb.
		\end{align*}
	\item Choose $a\in A$ and $b\in B$. We have
		\begin{align*}
			a(R^{-1})^{-1}b
				&\iff	bR^{-1}a	\\
				&\iff	aRb.
		\end{align*}
	\item Fix $a\in A, c\in C$ and observe that
		\begin{align*}
			c(S\circ R)^{-1}a
				&\iff	a(S\circ R)c					\\
				&\iff	\exists b\in B : aRb\wedge bSc			\\
				&\iff	\exists b\in B : bR^{-1}a\wedge cS^{-1}b	\\
				&\iff	c(R^{-1}\circ S^{-1})a.
		\end{align*}
	\item Let $a\in A$ and $d\in D$. We have
		\begin{align*}
			a(T\circ S)\circ Rd
				&\iff	\exists b\in B : aRb \wedge b(T\circ S)d			\\
				&\iff	\exists b\in B : aRb \wedge (\exists c\in C : bSc\wedge cTd)	\\
				&\iff	\exists b\in B : \exists c\in C : aRb \wedge bSc\wedge cTd	\\
				&\iff	\exists c\in C : (\exists b\in B :aRb \wedge bSc)\wedge cTd	\\
				&\iff	aT\circ(S\circ R)d.
		\end{align*}
	\item Fix $a\in A$ and observe that
		\begin{align*}
			a\in\Dom R
				&\iff	\exists b\in B : aRb		\\
				&\iff	\exists b\in B : bR^{-1}a	\\
				&\iff	a\in\Rng R^{-1}.
		\end{align*}
	\item Fix $b\in B$ and observe that
		\begin{align*}
			b\in\Rng R
				&\iff	\exists a\in A : aRb		\\
				&\iff	\exists a\in A : bR^{-1}a	\\
				&\iff	a\in\Dom R^{-1}.
		\end{align*}
	\item Let $A=\{1,2,3\}$ and define the relations
		\begin{align*}
			R	&=	\{(1,1),(1,2),(2,1)(2,2),(3,3)\}	\\
			S	&=	\{(1,1),(2,2),(2,3),(3,2),(3,3)\}.
		\end{align*}
		Note that both $R$ and $S$ are equivalence relations on $A$, while
		\[
			S\circ R = A^2\backslash \{(3,1)\}
		\]
		is neither transitive nor symmetric. \label{quest:equivalence_relations_not_closed_under_composition}
	\item Fix $a,b,c\in A$.

		If $aRa$ then $a(R\circ R)a$, whence $(R\circ R)$ is reflexive.

		Suppose that $a(R\circ R)b$ and $b(R\circ R)c$. Transitivity of $R$ provides $aRb$ and $bRc$, from which we obtain $a(R\circ R)c$. This establishes the transitivy of $R\circ R$.

		Now suppose that $a(R\circ R)b$ and $b(R\circ R)a$. As above, the transitivity of $R$ yields $aRb$ and $bRa$, and antisymmetry provides $a=b$. We conclude that $R\circ R$ is antisymmetric.
	\item Define the relations $R$ and $S$ on $A=\Z$ by
		\begin{align*}
			mRn	&\iff	m\leq n		\\
			mSn	&\iff	m\geq n.
		\end{align*}
		Observe that both $R$ and $S$ are partial orders, while
		\[
			S\circ R = \Z\times\Z
		\]
		is not antisymmetric. \label{quest:partial_orders_not_closed_under_composition}
	\item Let $A=\{1,2,3\}$ and let $R$ and $S$ be the relations of Question \ref{quest:equivalence_relations_not_closed_under_composition}. We have
		\begin{align*}
			S\circ R
				&=	A^2\backslash \{(3,1)\}		\\
				&\neq	A^2\backslash \{(1,3)\}		\\
				&=	R\circ S.
		\end{align*}
	\item Let $A$, $R$, and $S$ be given as in Question \ref{quest:partial_orders_not_closed_under_composition}.
\end{enumerate}








\end{document}





































