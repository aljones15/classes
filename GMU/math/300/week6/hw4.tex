\documentclass{article}
  \usepackage{amsfonts}
  \usepackage{amsthm}
  \usepackage{mathtools}
\begin{document}
    MATH-300 \hfill Andrew Jones
  \begin{center}
  {\huge Worksheet 4}
  \end{center}
  Let $R$ be a relation from $A$ to $B$, let $S$ be a relation from $B$ to $C$, and let $T$ be a relation from $C$ to $D$. \\ Prove the following statements. \\
  \begin{enumerate}
    % 1
    \item $I_A \circ R = R$
      \begin{proof}
        Let $a\in A$ and $b\in B$ and observe:
        \begin{center}
          $a(I_a \circ R)b \iff \exists a^`\in A: a = a^` \wedge a^`Rb$ \\
          $\iff aRb$
        \end{center}
      \end{proof}
    % 2
    \item $R \circ I_A = R$
      \begin{proof}
        Let $a\in A$ and $b\in B$ and observe:
        \begin{center}
          $a(R \circ I_a)b \iff \exists b^{'}\in B: b = b^{'} \wedge aRb^{'}$ \\
          $\iff aRb$
        \end{center}
      \end{proof}
    % 3
    \item $(R^{-1})^{-1} = R$
      \begin{proof}
        Assume the relation R has an inverse and let $a\in A$ and $b\in B$:
        \begin{center}
          $a(R^{-1})^{-1}b \iff \exists b^{'}\in B: b = b^{'} \wedge b^{'}R^{-1}a$ \\
          $\iff b(R^{-1})^{-1}a$
        \end{center}
      \end{proof}
    % 4
    \item $(S \circ R)^{-1} = R^{-1}\circ S^{-1}$
      \begin{proof}
        
      \end{proof}
    % 5
    \item $(T \circ S) \circ R = T \circ (S \circ R)$
      \begin{proof}
        
      \end{proof}
    % 6
    \item $Dom R = Rng R^{-1}$
      \begin{proof}
        
      \end{proof}
    % 7
    \item $Rng R = Dom R^{-1}$
      \begin{proof}
        
      \end{proof}
  \end{enumerate}
  For Question 8–10, suppose that $A = B = C$.
  \begin{enumerate} \setcounter{enumi}{7}
    % 8
    \item If $R$ and $S$ are equivalence relations, then $S \circ R$ is an equivalence relation.
      \begin{proof}
        
      \end{proof}
    % 9
    \item If $R$ is a partial order, then $R \circ R$ is a partial order.
      \begin{proof}
        
      \end{proof}
    % 10
    \item If $R$ and $S$ are partial orders, then it is not generally true that $S \circ R$ is a partial order.
      \begin{proof}
        
      \end{proof}
  \end{enumerate}
  \textbf{Bonus Questions}
  Give an example of two relations $R$ and $S$ on a set $A$ such that \\
  \begin{enumerate} \setcounter{enumi}{10}
    % 11
    \item $R \circ S \neq S \circ R$.
      \begin{proof}
        
      \end{proof}
    % 12
    \item $S \circ R$ is an equivalence relation, but neither $R$ nor $S$ is an equivalence relation.
      \begin{proof}
        
      \end{proof}
   \end{enumerate}
\end{document}
