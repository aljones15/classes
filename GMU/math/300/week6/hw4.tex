\documentclass{article}
  \usepackage{amsfonts}
  \usepackage{amsthm}
  \usepackage{mathtools}
\begin{document}
    MATH-300 \hfill Andrew Jones
  \begin{center}
  {\huge Worksheet 4}
  \end{center}
  Let $R$ be a relation from $A$ to $B$, let $S$ be a relation from $B$ to $C$, and let $T$ be a relation from $C$ to $D$. \\ Prove the following statements. \\
  \begin{enumerate}
    \item $I_A \circ R = R$
      \begin{proof}
        
      \end{proof}
    \item $R \circ I_A = R$
    \item $(R^{-1})^{-1} = R$
    \item $(S \circ R)^{-1} = R^{-1}\circ S^{-1}$
    \item $(T \circ S) \circ R = T \circ (S \circ R)$
    \item $Dom R = Rng R^{-1}$
    \item $Rng R = Dom R^{-1}$
  \end{enumerate}
  For Question 8–10, suppose that $A = B = C$.
  \begin{enumerate} \setcounter{enumi}{7}
    \item If $R$ and $S$ are equivalence relations, then $S \circ R$ is an equivalence relation.
    \item If $R$ is a partial order, then $R \circ R$ is a partial order.
    \item If $R$ and $S$ are partial orders, then it is not generally true that $S \circ R$ is a partial order.
  \end{enumerate}
  \textbf{Bonus Questions}
  Give an example of two relations $R$ and $S$ on a set $A$ such that \\
  \begin{enumerate} \setcounter{enumi}{10}
    \item $R \circ S \neq S \circ R$.
    \item $S \circ R$ is an equivalence relation, but neither $R$ nor $S$ is an equivalence relation.
   \end{enumerate}
\end{document}
