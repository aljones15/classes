\documentclass{article}
  \usepackage{amsfonts}
  \usepackage{amsthm}
  \usepackage{mathtools}
\begin{document}
    MATH-300 \hfill Andrew Jones
  \begin{center}
  {\huge Worksheet 4}
  \end{center}
  Let $R$ be a relation from $A$ to $B$, let $S$ be a relation from $B$ to $C$, and let $T$ be a relation from $C$ to $D$. \\ Prove the following statements. \\
  \begin{enumerate}
    % 1
    \item $I_A \circ R = R$
      \begin{proof}
        Let $a\in A$ and $b\in B$ and observe:
        \begin{align*}
          a(I_a \circ R)b &\iff \exists a^`\in A: a = a^` \wedge a^`Rb \\
          &\iff aRb
        \end{align*}
      \end{proof}
    % 2
    \item $R \circ I_A = R$
      \begin{proof}
        Let $a\in A$ and $b\in B$ and observe:
        \begin{align*}
          a(R \circ I_a)b &\iff \exists b^{'}\in B: b = b^{'} \wedge aRb^{'} \\
          &\iff aRb
        \end{align*}
      \end{proof}
    % 3
    \item $(R^{-1})^{-1} = R$
      \begin{proof}
        Assume the relation R has an inverse and let $a\in A$ and $b\in B$:
        \begin{align*}
          a(R^{-1})^{-1}b &\iff bR^{-1}a \\
          &\iff b(R^{-1})^{-1}a
        \end{align*}
      \end{proof}
    % 4
    \item $(S \circ R)^{-1} = R^{-1}\circ S^{-1}$
      \begin{proof}
        Let $a\in A$, $b\in B$, and $c\in C$
        \begin{align*}
          c(S \circ R)^{-1}a &\iff \exists c^{'}\in C: c = c^{'} \wedge c^{'}S^{-1}b \\
          &\iff \exists b^{'}\in B: b=b^{'} \wedge b^{'}R^{-1}a \\
          &\iff c(R^{-1} \circ S^{-1})a
        \end{align*}
      \end{proof}
    % 5
    \item $(T \circ S) \circ R = T \circ (S \circ R)$
      \begin{proof}
        Fix $a\in A$, $b\in B$, $c\in C$, and $d\in D$
        \begin{align*}
          a(T\circ S)\circ Rd &\iff \exists a^{'}\in A: a^{'}=a \wedge a^{'}Rb \\
          &\iff \exists b^{'}\in B: b^{'}=b \wedge b^{'}Sc \\
          &\iff \exists c^{'}\in C: c^{'}=c \wedge c^{'}Td \\
          &\iff aT\circ(S\circ R)d
        \end{align*}
      \end{proof}
    % 6
    \item $Dom\,R = Rng\,R^{-1}$
      \begin{proof} ($\subseteq$)
        Suppose R is invertible and fix $r\in Rng\,R^{-1}$. By definition of inverse it follows that $r\in Dom\,R$.
      \end{proof}
      \begin{proof} ($\supseteq$)
Fix $d\in Dom\,R$. By definition of domain it follows that $d \in Rng\,R^{-1}$.
      \end{proof}
    % 7
    \item $Rng\,R = Dom\,R^{-1}$
      \begin{proof} ($\supseteq$)
        Suppose R is invertible and $r\in Rng\,R$. By the definition of inverse $r\in Dom\,R^{-1}$.
      \end{proof}
      \begin{proof} ($\subseteq$)
Fix $d\in Dom\,R^{-1}$. By the defintion of domain it follows that $d\in Rng\,R$.
      \end{proof}
  \end{enumerate}
  For Question 8–10, suppose that $A = B = C$.
  \begin{enumerate} \setcounter{enumi}{7}
    % 8 reflexive, transitive, and symmetric
    \item If $R$ and $S$ are equivalence relations, then $S \circ R$ is an equivalence relation.
      \begin{proof}
        Suppose $R$ is an equivalence relation from $A$ to $B$ and $S$ is an equivalence relation from $B$ to $C$ and $A = B = C$.
        \begin{align*}
          S \circ R &\iff \forall a\in A : aSa \wedge aRa \\
          &\iff \forall a,b,c\in A: (aSb \wedge bSc) => aSa \wedge (aRb \wedge bRc) => aRc \\
            &\iff \forall a,b\in A: (aSb \wedge bSa) \wedge (aRb \wedge bRa) 
        \end{align*}
      \end{proof}
    % 9 reflexive, transitive, antisymmetric
    \item If $R$ is a partial order, then $R \circ R$ is a partial order.
      \begin{proof}
        Suppose R is a partial order from $A$ to $B$ and $A = B$
        \begin{align*}
          R &\iff \forall a\in A: aRa \\
            &\iff \forall a,b, c\in A: (aRb \wedge bRc) => aRc \\
            &\iff \forall a,b\in A: (aRb \wedge bRa) => a = b \\
              &\iff R \circ R
        \end{align*}
      \end{proof}
    % 10 reflexive, transitive, antisymmetric
    \item If $R$ and $S$ are partial orders, then it is not generally true that $S \circ R$ is a partial order.
      \begin{proof}
        
      \end{proof}
  \end{enumerate}
  \textbf{Bonus Questions}
  Give an example of two relations $R$ and $S$ on a set $A$ such that \\
  \begin{enumerate} \setcounter{enumi}{10}
    % 11
    \item $R \circ S \neq S \circ R$.
      \begin{proof}
        
      \end{proof}
    % 12
    \item $S \circ R$ is an equivalence relation, but neither $R$ nor $S$ is an equivalence relation.
      \begin{proof}
        
      \end{proof}
   \end{enumerate}
\end{document}
