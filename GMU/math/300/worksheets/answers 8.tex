\documentclass{article}
\usepackage{amsmath,amssymb,amsthm}
\usepackage{fullpage}

\newcommand{\R}{\mathbb{R}}
\newcommand{\Z}{\mathbb{Z}}
\newcommand{\N}{\mathbb{N}}

\renewcommand{\epsilon}{\varepsilon}




\title{Worksheet 8 Answer Key}
\author{}\date{}



\begin{document}\maketitle

\begin{enumerate}
	\item 
		\begin{proof}
			Fix $x\in\R$ and let $\epsilon>0$. By the continuity of $f$ and $g$, there are $\delta_1$ and $\delta_2>0$ such that
			\[
				|x-y|<\delta_1 \implies |f(x)-f(y)| < \frac{\epsilon}{2}
			\]
			and
			\[
				|x-y|<\delta_2 \implies |g(x)-g(y)| < \frac{\epsilon}{2}.
			\]
			Let $\delta$ be the lesser of $\delta_1$ and $\delta_2$ and suppose that $|x-y|<\delta$. We have
			\begin{align*}
				|(f+g)(x)-(f+g)(y)|
					&=	|f(x)+g(x)-f(y)-g(y)|			\\
					&\leq	|f(x)-f(y)| + |g(x)-g(y)|		\\
					&<	\epsilon.
			\end{align*}
		\end{proof}
	\item No.
	\item No, $(f(x_i))_i$ is not necessarily divergent.
		\begin{proof}
			Let $X=\R$ with the standard metric $d_\R(x,y)=|x-y|$, let $Y=\{0\}$, and let $f:\R\to\{0\}$ be the constant function. Observe that the sequence $(x_i)_\subset\R$ given by $x_i=i$ diverges while the constant sequence $f(x_i)=0$ converges.
		\end{proof}
	\item 
		\begin{proof}
			Let $x\in X$ and suppose that $\epsilon>0$. If $y\in X$ with $d_X(x,y)<\epsilon$, then $d_Y(f(x),f(y)))=d_X(x,y)<\epsilon$.
		\end{proof}
	\item 
		\begin{proof}
			Suppose that $(x_i)_i\subset X$ converges to $x\in X$. Let $\epsilon>0$ and choose $N\geq 0$ so that $d(x_i,x)<\frac{\epsilon}{2}$ for all $i\geq N$. Let $m,n\geq N$ and observe that
			\[
				d(x_n,x_m) \leq d(x_m,x) + d(x,x_m) < \epsilon.
			\]
		\end{proof}
	\item Let $X=\R_{>0}$ with the standard metric $d(x,y)=|x-y|$. The sequence $(x_i)_i$ with $x_i = \frac{1}{i}$ does not converge to any element of $\R_{>0}$.
	\item 
		\begin{proof}
			Fix $x\in X$ and let $\epsilon>0$. Since $f$ is uniformly continuous, there is a $\delta>0$ for which
			\[
				d_X(x,y) < \delta \implies d_Y(f(x),f(y))<\epsilon.
			\]
		\end{proof}
	\item
		\begin{proof}
			Let $x\in\R$ and $\epsilon>0$. If $y\in\R$ satisfies $|x-y|<\min\big(1,\frac{\epsilon}{1+2|x|}\big)$, then
			\begin{align*}
				|f(x)-f(y)|
					&=	|x^2-y^2|					\\
					&=	|x-y|\,|x+y|				\\
					&=	|x-y|\,|2x-x+y|				\\
					&\leq	|x-y|\,\big(2|x| + |x-y| \big)		\\
					&<	\frac{\epsilon}{1+2|x|}  \big(2|x|+1\big)	\\
					&=	\epsilon.
			\end{align*}
		\end{proof}
	\item 
		\begin{proof}
			Put $\epsilon=1$ and let $\delta>0$. Put $x=\frac{1}{\delta}+\frac{\delta}{2}$ and $y=\frac{1}{\delta}$. Observe that $|x-y|<\delta$ while
			\begin{align*}
				|f(x)-f(y)|
					&=	\Big|\Big(\frac{1}{\delta}+\frac{\delta}{2}\Big)^2 - \Big(\frac{1}{\delta}\Big)^2\Big|		\\
					&=	1 + \frac{\delta^2}{4}										\\
					&\geq	\epsilon.
			\end{align*}
		\end{proof}
	\item 
		\begin{proof}
			Fix $\epsilon>0$. If $x,y\in X$ satisfy $d_X(x,y)< \epsilon$, then
			\[
				d_Y(f(x),f(y)) < d_X(x,y) < \epsilon.
			\]
		\end{proof}
\end{enumerate}








\end{document}





































