\documentclass[12pt, openany, closeany]{book}
  \usepackage{amsthm, amsmath, amsfonts}
  \usepackage[Sonny]{fncychap}
  \usepackage[a5paper, margin=12mm]{geometry}
  \usepackage[T1]{fontenc}
  \usepackage{mathptmx}
  \usepackage{titlesec}
  \usepackage[symbol]{footmisc}
  \newtheorem*{theorem}{Theorem}
  \newtheorem{lemma}{Lemma}
  \newtheoremstyle{corollary}%
  {\topsep}% measure of space to leave above the theorem. E.g.: 3pt
  {\topsep}% measure of space to leave below the theorem. E.g.: 3pt
  {\itshape}% name of font to use in the body of the theorem
  {0pt}% measure of space to indent
  {\bfseries}% name of head font
  {}% punctuation between head and body
  {  }% space after theorem head; " " = normal interword space
  {\thmname{#1}\thmnumber{ #2}\textnormal{\thmnote{ (#3)}}}
  
  \theoremstyle{corollary}
  \newtheorem*{corollary}{Corollary}
\begin{document}
  \chapter*{
    \begin{center}
    $\xi$2. THE THEOREM \\ OF SARD AND BROWN
    \end{center}
  }
  In general, it is too much to hope that the set of critical values of a \\ smooth map be finite. But this set will be ``small,'' in the sense indicated\\ by the next theorem, which was proved by A. Sard in 1942 following\\ earlier work by A.P. Morse, (References [30], [24].)
  \begin{theorem}
  $let f : U -> R^n$ be a smooth map, defined on an open set $U \subset R^m$, and let
  \begin{center}
  $C = \{x\in U|\ rank\ df_{x} < n \}$
  \end{center}
  Then the image $f(C) \subset R^n$ has Lebesgue measure zero \footnote[1]{In other words, given any $\xi > 0$, it is possible to cover $f(C)$ by a sequence of cubes in $R^n$ having total n-dimensional volume less than $\xi$ }
  \end{theorem}
  Since a set of measure zero cannot contain any nonvacuous open set, \\
  it follows that the complement $R^n - f(C)$ must be everywhere dense\footnote[2]{Proved by Arthur B. Brown in 1935. This result was rediscovered by Dubovickil in 1953 and by Thorn in 1954. (References [5], [8], [36].)} in $R^n$. \\
  The proof will be given in $\xi3$. It is essential for the proof that $f$ should \\
  have many derivatives. (Compare Whitney [38].) \\
  We will be mainly interested in the case $m\geq n$. If $m < n$, then \\
  clearly $C = U$; hence the theorem says simply that $f(U)$ has measure \\
  zero. \\
  More generally consider a smooth map $f : M => N$, from a manifold \\
  of dimension $m$ to a manifold of dimension $n$. Let $C$ be the set of all \\
  $x\in \mathbb{M}$ such that
  \begin{align*}
    df_{x} : TM_{x}\to TN_{f(x)}
  \end{align*}
  \newpage
  Regular values \hfill 11 \\ \\
  has rank less than $n$ (i.e. is not onto). Then $C$ will be called the set \\
  of critical points, $f(C)$ the set of critical values, and the complement \\
  $N - f(C)$ the set of regular values of $f$. (This agrees with our previous \\
  definitions in the case $m = n$.) Since $M$ can be covered by a countable \\
  collection of neighborhoods each diffeomorphic to an open subset of \\
  $R^m$, we have:
  \begin{corollary}
  (A. B. Brown). The set of regular values of a smooth map $f: M\to N$ is everywhere dense in $N$.
  \end{corollary}
  In order to exploit this corollary we will need the following:
  \begin{lemma}
  If $f: M\to N$ is a smooth map between manifolds of dimension $m\geq n$, and if $y\in N$ is a regular value, then the set $f^{-1}(y) \subset M$ is a smooth manifold of dimension $m - n$.
  \end{lemma}
  \begin{proof} Let $x\in f^{-1}(y)$. Since $y$ is a regular value, the derivative $df_{x}$ must map $TM_{x}$ 
  \end{proof}
  
\end{document}
