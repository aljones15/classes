\documentclass[12pt, openany, closeany]{book}
  \usepackage{amsthm, amsmath, amsfonts}
  \usepackage[Sonny]{fncychap}
  \usepackage[a5paper, margin=12mm]{geometry}
  \usepackage[T1]{fontenc}
  \usepackage{mathptmx}
  \usepackage{titlesec}
  \usepackage[symbol]{footmisc}
  \usepackage{tikz}
  \newtheorem*{theorem}{Theorem}
  \newtheorem{lemma}{Lemma}
  \newtheoremstyle{corollary}%
  {\topsep}% measure of space to leave above the theorem. E.g.: 3pt
  {\topsep}% measure of space to leave below the theorem. E.g.: 3pt
  {\itshape}% name of font to use in the body of the theorem
  {0pt}% measure of space to indent
  {\bfseries}% name of head font
  {}% punctuation between head and body
  {  }% space after theorem head; " " = normal interword space
  {\thmname{#1}\thmnumber{ #2}\textnormal{\thmnote{ (#3)}}} 
  \theoremstyle{corollary}
  \newtheorem*{corollary}{Corollary}
  \newtheorem*{definition}{DEFINITION}
  \pagestyle{empty}
  \begin{document}
  \chapter*{
    \begin{center}
    $\xi$2. THE THEOREM \\ OF SARD AND BROWN
    \end{center}
  }
  In general, it is too much to hope that the set of critical values of a smooth map be finite. But this set will be ``small,'' in the sense indicated by the next theorem, which was proved by A. Sard in 1942 following earlier work by A.P. Morse, (References [30], [24].)
  \begin{theorem}
  $let f : U \to R^n$ be a smooth map, defined on an open set $U \subset R^m$, and let
  \begin{center}
  $C = \{x\in U|\ rank\ df_{x} < n \}$
  \end{center}
  Then the image $f(C) \subset R^n$ has Lebesgue measure zero \footnote[1]{In other words, given any $\xi > 0$, it is possible to cover $f(C)$ by a sequence of cubes in $R^n$ having total n-dimensional volume less than $\xi$ }
  \end{theorem}
  Since a set of measure zero cannot contain any nonvacuous open set, \\
  it follows that the complement $R^n - f(C)$ must be everywhere dense\footnote[2]{Proved by Arthur B. Brown in 1935. This result was rediscovered by Dubovickil in 1953 and by Thorn in 1954. (References [5], [8], [36].)} in $R^n$. \\
  The proof will be given in $\xi3$. It is essential for the proof that $f$ should \\
  have many derivatives. (Compare Whitney [38].) \\
  We will be mainly interested in the case $m\geq n$. If $m < n$, then \\
  clearly $C = U$; hence the theorem says simply that $f(U)$ has measure \\
  zero. \\
  More generally consider a smooth map $f : M => N$, from a manifold \\
  of dimension $m$ to a manifold of dimension $n$. Let $C$ be the set of all \\
  $x\in \mathbb{M}$ such that
  \begin{align*}
    df_{x} : TM_{x}\to TN_{f(x)}
  \end{align*}
  Regular values \hfill 11 \\ \\
  has rank less than $n$ (i.e. is not onto). Then $C$ will be called the set \\
  of critical points, $f(C)$ the set of critical values, and the complement \\
  $N - f(C)$ the set of regular values of $f$. (This agrees with our previous \\
  definitions in the case $m = n$.) Since $M$ can be covered by a countable \\
  collection of neighborhoods each diffeomorphic to an open subset of \\
  $R^m$, we have:
  \begin{corollary}
  (A. B. Brown). The set of regular values of a smooth map $f: M\to N$ is everywhere dense in $N$.
  \end{corollary}
  In order to exploit this corollary we will need the following:
  \begin{lemma}
  If $f: M \to N$ is a smooth map between manifolds of dimension $m\geq n$, and if $y\in N$ is a regular value, then the set $f^{-1}(y) \subset M$ is a smooth manifold of dimension $m - n$.
  \end{lemma}
  \begin{proof} Let $x\in f^{-1}(y)$. Since $y$ is a regular value, the derivative $df_{x}$ must map $TM_{x}$ onto $TN_{y}$. The null space $\Re \subset TM_{x}$ of $df_{x}$ will therefore be an $(m - n)$-dimensional vector space.
  \end{proof} 
    If $M \subset R^{k}$ choose a linear map $L : R^K \to R^{m-n}$ that is nonsingular on this subspace $\Re \subset TM_{x} \subset R^k$. Now define
    \begin{align*}
    F : M \to N X R^{m-n}
    \end{align*}
    by $F(\xi) = (f(\xi), L(\xi))$. The derivative $dF_{x}$ is clearly given by the formula
    \begin{align*}
      dF_{x}(v) = (df_{x}(v), L(v)).
    \end{align*}
    Thus $dF_{x}$ is nonsigular. Hence $F$ maps some neighborhood $U$ of $x$ diffeomorphically onto a neighborhood $V$ of $(y, L(x))$. Note that $f^{-1}(y)$ corresponds, under $F$, to hyperplane $y X R^{m-n}\cap V$. This proves that $f^{-1}(y)$ is a smooth manifold of dimension $m - n$. 
    As an example we can give an easy proof that the unit sphere $S^{n-1}$ is a smooth manifold. Consider the function $f : R^m \to R$ defined by
    \begin{align*}
      f(x) = x^2_{1} + x^2_{2} + ... + x^2_{m}.
    \end{align*}
    Any $y \neq 0$ is a regular value, and the smooth manifold $f^{-1}(1)$ is the unit sphere.
    If $M'$ is a manifold which is contained in $M$, it has already been noted that $TM'_{x}$ for $x\in M'$ is a subscape of $TM_{x}$ for $x\in M'$. The orthogonal complement of $TM'_{x}$ in $TM_{x}$ is then a vector space of dimension $m - m'$ called the \emph{the space of normal vectors to M' in M at} $x$.
    In particular let $M' = f^{-1}(y)$ for a regular value $y$ of $f: M \to N$.
  \begin{lemma} The null space of $df_{x} : TM_{x} \to TN_{y}$ is precisely equal to the tangent space $TM'_{x} \subset TM_{x}$ of the submanifold $M' = f^{-1}(y)$. Hence $df_{x}$ maps the orthogonal complement of $TM'_{x}$ isomorphically onto $TN_{y}$.

  \end{lemma}
  \begin{proof} From the diagram \\
    \begin{center}
    \begin{tikzpicture}
      \draw[thick,->] (4,4) -- (4,0) node[anchor=north west] {y};
      \draw[thick,->] (4,0) -- (8,0) node[anchor=north west] {N};
      \draw[thick,->] (4,4) -- (8,4) node[anchor=north west] {M};
      \draw[thick,->] (8,4) -- (8,0);
    \end{tikzpicture}
    \end{center}
    we see that $df_{x}$ maps the subspace $TM'_{x} \subset TM_{x}$ to zero. Counting dimensions we see that $df_{x}$ \emph{maps the space of normal vectors to M' isomorphically onto} $TN_{y}$.
  \end{proof}
  \section{Manifolds with Boundary}
  The lemmas above can be sharpened so as to apply to a map defined on a smooth "manifold with boundary." Consider first the closed half-space
  \begin{align*}
    H^m = {(x_{1}, ..., x_{m})\in R^m | x_{m}\geq 0}.
  \end{align*}
  The boundary $\partial H^m$ is defined to be the hyperplance $R^{m-1} \times 0 \subset R^M$.
  \begin{definition}
    A subset $X \subset R^k$ is called a \emph{smooth m-manifold with boundary} if each $x\in X$ has a neighbohood $U \cup X$ diffeomorphic to an open subset $V\cup H^m$ of $H^m$. The \emph{boundary} ${\partial X}$ is the set of all points in $X$ which correspond to points of $\partial H^m$ under such a diffeomorphism.
  \end{definition}
  It is not hard to show that $\partial x$ is a well-defined smooth manifold of dimension $m - 1$. The \emph{interior} $X - \partial X$ is a smooth manifold of dimension $m$.
  The tangent space $TX_{z}$ is defined just as in $\xi 1$, so that $TX_{x}$ is a full $m$-dimensional vector space, even if $x$ is a boundary point.
  Here is one method for generating examples. Let $M$ be a manfiold without boundary and let $g : M \to R$ have $0$ as regular value.
  \begin{lemma}
    The set of $x$ in $M$ with $g(x) \geq 0$ is a smooth manifold, with boundary equal to $g^{-1}(0)$.
  \end{lemma}
  The proof is just like the proof of Lemma 1.
\end{document}
